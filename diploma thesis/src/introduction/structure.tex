\section{Structure of this Document}

We begin with an introduction to some fundamental topics in \autoref{fundamentals}.
First, we explain bootstrapping, followed by a general overview of provisioning with some details on TOSCA\footnote{\url{https://www.oasis-open.org/committees/tc_home.php?wg_abbrev=tosca}}~\autocite{tosca:spec} and OpenTOSCA.
We explain the concept of cloud computing and describe Amazon's cloud platform.
We also present the basics of service oriented architecture, workflows, and workflow management systems.
Finally, we describe the SimTech project as well as the SimTech SWfMS.

In \autoref{previous} we present previous work on the subject of this diploma thesis.
First, we summarize the paper that build the foundation of this diploma thesis.
Then, we discuss a previous diploma thesis that extended parts of this paper.
In \autoref{related} we also present some related work.
We list the requirements that were given for this diploma thesis in \autoref{requirements}.
We also explain some additional constraints that we introduced.

We present the design of the bootware in \autoref{design}.
First, we discuss component division, followed by the integration into existing modeler applications.
Next, we select an external communication mechanism.
We describe the extensibility mechanism, followed by the different kinds of plugins.
We also discuss internal communication, the context object, the web service interface, and the instance store.
Then, we describe the execution flow and the use of finite state machines, before the final bootware architecture is presented.
We also present a step by step description of the whole bootstrapping process in \autoref{process}.

In \autoref{implementation} we present details on the implementation of the bootware.
We describe the integration into the SimTech Modeler with an Eclipse plugin.
We also explain the bootware core library.
Then, we select the plugin framework, publish subscribe library, and state machine library that we will use for the implementation.
We also describe the context object and the web service operation.
Then, we give an overview over some plugins we implemented.
In \autoref{future} we list some possibilities for future improvement.
We summarize the previous chapters in \autoref{conclusion}, before presenting a conclusion.
