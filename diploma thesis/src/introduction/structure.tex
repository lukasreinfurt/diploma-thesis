\section{Structure of this Document}

\autoref{fundamentals} introduces some fundamental topics.
It describes the SimTech project and the SimTech SWfMS, which initiated this diploma thesis.
It also gives a short introduction into bootstrapping.
Then, cloud computing in general is explained, followed by a closer look at the Amazon Web Services, which are used in this diploma thesis.
Finally, provisioning in general, as well as with TOSCA and OpenTOSCA is described.

\autoref{related} presents work related to this diploma thesis.
First, the paper that builds the foundation of this diploma thesis is discussed, followed by a previous diploma thesis that extended parts of this paper.
Then, another architecture for automatic provisioning of cloud services is presented.

\autoref{requirements} lists the requirements that were given for this diploma thesis.
It also introduces some additional constrains that were introduced during the writing of this diploma thesis.

\autoref{design} presents the design of the bootware.
First, the component division is discussed, followed by the integration into existing modeler applications.
Then, an external communication mechanism is selected.
The extensibility mechanism is described next, followed by the different kinds of plugins.
Internal communication, the context object, the web service interface, and the instance store are also discussed.
The execution flow and the use of finite state machines is described, before the final bootware architecture is presented.

\autoref{process} describes the whole boostrapping process step by step.

\autoref{implementation} presents details on the implementation of the bootware.
The reasons for the selection of the plugin framework and the PubSub and state machine libraries are discussed.
Next, the integration into the SimTech Modeler using an Eclipse plugin is described.
Then, the implementation of the context object and the web service interface are presented.

\autoref{conclusion} summarizes the previous chapters.
It describes the work that has been done and also the work that still needs to be done, areas that could be improved, and future work that might be useful.
Finally, a conclusion is presented.
