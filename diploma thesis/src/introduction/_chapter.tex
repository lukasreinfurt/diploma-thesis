\chapter{Introduction}

Workflow technology and the service based computing paradigm were mostly used in a business context until now.
But slowly they are extended to be used in other fields, such as eScience, where business centric assumptions that where previously true are not reasonable anymore.
One of these assumptions is that services should run continuously.
This made sense in large enterprises where those services are used every day.
Science, on the other hand, often takes a more dynamic approach, where certain services, for example for simulation purposes, are only used at certain times.
In those cases, it would make more sense to dynamically provision services only when they are needed.
To provision those service, provisioning engines might be used, but these also have to be set up first.
This creates the need for a bootstrapping mechanism that can deploy provisioning engines when needed.

\section{Task of this Diploma Thesis}

The task of this diploma thesis is to design a lightweight bootstrapping system that can kick off dynamic provisioning in cloud environments.
It should be able to provision various provisioning engines in all kinds of cloud environments.
The provisioning engines then handle the actual provisioning of required workflow systems and services.
A managing component that keeps track of provisioned environments is also part of this system.

Support for different cloud environments and provisioning engines should be achieved through means of software engineering.
A functioning prototype that supports Amazon as cloud environment and TOSCA as provisioning engine should be implemented.

\section{Structure of this Document}

...

