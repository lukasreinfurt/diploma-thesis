\section{Plugins}
\label{implementation:plugins}

Now, we will describe the implementation of a few plugins.
We implemented an infrastructure plugin that can create and remove EC2 instances in Amazon's cloud.
We created a connection plugin that allows the bootware to connect to a remote system via SSH and then execute commands on, or upload files to this system.

\subsection{AWS EC2 Plugin}

This infrastructure plugin allows the bootware to create and remove EC2 instances in Amazon's cloud.
It uses the AWS SDK for Java\footnote{\url{http://aws.amazon.com/sdkforjava/}} to implement this functionality.
This SDK specifies a specific set of action that have to be taken to accomplish this, which we map onto the operations defined by each infrastructure plugin, i.e. init, shutdown, deploy, and undeploy, as described in \autoref{design:plugins}.
\autoref{image:awsplugin} shows a simplified overview of these actions and how they map onto the infrastructure plugin operations.

\begin{figure}[!htbp]
	\centering
	\includegraphics[resolution=600]{implementation/assets/aws_plugin}
	\caption{The operations implemented by the AWS EC2 plugin.}
	\label{image:awsplugin}
\end{figure}

The init operation, shown on the left of \autoref{image:awsplugin}, which is called once when the plugin is loaded, creates a client instance, which is an object on which all the following actions will be called.

As we can see in the deploy operation in \autoref{image:awsplugin}, we first have to create a security group\footnote{\url{http://docs.aws.amazon.com/AWSEC2/latest/UserGuide/using-network-security.html}}.
Security groups are essentially virtual firewalls that allow or deny traffic to and from all EC2 instances associated with it.
EC2 instances have to be associated with a security group, so we create one.
In the next step we open all ports in this security group that we later want to use for communication.
We also have to create a SSH key pair and retrieve the private key, which we later use when we connect to this EC2 instance via SSH.
In the last step we create the actual EC2 instance.
Once it is up and running, the deploy operation is finished and returns an instance object which contains the URL where the EC2 instance can be reached, as well as the private key.

The undeploy operation reverses the deploy operation.
First, it terminates the EC2 instance.
Once the instance is stopped, the key pair and the security group that were created earlier are removed.
We don't have to close the ports we opened, since they are part of the security group and don't exist anymore once the security group is removed.
After this, the EC2 instance created earlier is successfully removed.
There are no further actions necessary during the shutdown operation, but for safety we call the undeploy operation, in case it wasn't called earlier.
