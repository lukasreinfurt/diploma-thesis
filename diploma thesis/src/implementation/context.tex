\section{Context}
\label{implementation:context}

\vspace*{\baselineskip}
\begin{listing}[!htbp]
	\inputminted[
		label=context.xml,
		frame=topline,
		linenos,
		frame=lines,
		tabsize=2,
		framesep=0.3cm,
		fontsize=\small
	]{xml}{implementation/assets/context.xml}
	\caption{Sample context represented in XML.}
	\label{lst:context:sample}
\end{listing}

\autoref{lst:context:sample} shows an exemplary context generated by the bootware in XML form.
As we can see in line 2-4, it defines the resource, connection, and application plugins that should be used during the bootstrapping process by supplying the name of the plugin \textit{.jar}.
It can also contain a provision workflow middleware plugin, as can be seen in line 6-8.
This is optional and will only be used on the first request, when the remote bootware will also call a provisioning engine to provision the workflow middleware.
This is also where the service package reference in line 10-12 will be used, which points to the workflow middleware package that should be provisioned by the provisioning engine called by the provision workflow middleware plugin.
In line 14-35 we can also see the optional configuration list.
It contains configuration values that are passed to plugins if required.
In this case, it contains login credentials for Amazon's cloud, shown in line 19-26, which are used by the \textit{aws-ec2.jar} plugin to authenticate its requests made to Amazon.
In line 27-30 we can also see an instance type parameter.
This and other parameters will also be read by the \textit{aws-ec2.jar} plugin.
