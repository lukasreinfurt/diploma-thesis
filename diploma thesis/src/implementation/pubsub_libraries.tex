\subsection{PubSub Libraries}
\label{implementation:selecting:pubsub}

Many of the well know messaging middlewares offer support for PubSub, for example ActiveMQ\footnote{\url{http://activemq.apache.org}}, RabbitMQ\footnote{\url{http://www.rabbitmq.com}}, and ZeroMQ\footnote{\url{http://zeromq.org}}.
But, because we are looking for an internal communication mechanism only, all of these solutions are somewhat overpowered.
We do not have to worry about network problems, so we do not need guaranteed delivery or message queuing capabilities.
We also do not need persistence or transactional capabilities
We do not have to handle millions of subscribers or events, so high scalability is not a concern.
We do not even necessarily need asynchronous communication.
Instead, we need a lightweight in-memory solution.
Therefor we will ignore the middleware heavyweights and look for smaller PubSub libraries.

We have a few functional requirements that a library has to support for our use case.
These can be seen on the left-hand side of \autoref{table:pubsub_comparison}.
Weak references are an important feature because we have a lot of plugins that will register as listeners to the event bus.
These plugins can be removed at any time and weak references allow us to remove them without explicitly unregistering them from the event bus.
Instead of crashing, the event bus will just ignore references to listeners that do not exist anymore.
Even if we explicitly unregister all our plugins, weak references give us a safety net if we forget it at some point.

We also need support for an event hierarchy.
This allows us to model our events in a very fine grained modular fashion and organize them into logical groups.
It also allows listeners to react to a whole group of specific events or only to a small subset of such a group.
A filtering feature gives us even more control over what events a listener will react to.
It allows us to filter out specific events, for example by their content, to handle them differently, or to ignore them.

We also want event handlers to be invoked synchronously.
If an event is published, all event handlers for this event should be executed one after another until they are finished.
Only then should the program continue execution.
But asynchronous invocation might still be useful in some cases, so we also add it here.

\vspace*{\baselineskip}
\begingroup
	\centering
	\captionsetup{type=table}
	\begin{tabu}[!htbp]{rl|[0.5pt]ccccc}

		&
		& \multicolumn{5}{c}{\textit{PubSub Libraries}} \\[10pt]

		&
		& \begin{sideways} \textbf{EventBus\footnote{\url{http://eventbus.org/}\label{eventbus}}} \end{sideways}
		& \begin{sideways} \textbf{Guava Event Bus\footnote{\url{https://code.google.com/p/guava-libraries/wiki/EventBusExplained}\label{guava}}} \end{sideways}
		& \begin{sideways} \textbf{Simple Java Event Bus\footnote{\url{https://code.google.com/p/simpleeventbus/}\label{simpleeventbus}}} \end{sideways}
		& \begin{sideways} \textbf{MBassador\footnote{\url{https://github.com/bennidi/mbassador}\label{mbassasor}}} \end{sideways}
		& \begin{sideways} \textbf{Mycila PubSub\footnote{\url{https://github.com/mycila/pubsub}\label{mycilapubsub}}} \end{sideways} \\

		\tabucline[0.5pt]{2-7}

		% NO  = \ding{55}
		% YES = \ding{51}

		\multirow{5}{*}{\textit{functional}}

		& \textbf{Weak References}
		& \ding{51}    % eventbus
		& \ding{55}    % guava
		& \ding{51}    % simpleeventbus
		& \ding{51}    % mbassador
		& \ding{51} \\ % mycila

		& \textbf{Event Hierarchy}
		& ?            % eventbus
		& \ding{51}    % guava
		& ?            % simpleeventbus
		& \ding{51}    % mbassador
		& \ding{51} \\ % mycila

		& \textbf{Filtering}
		& \ding{51}    % eventbus
		& \ding{55}    % guava
		& \ding{51}    % simpleeventbus
		& \ding{51}    % mbassador
		& \ding{55} \\ % mycila

		& \textbf{Sync. Invocation}
		& \ding{51}    % eventbus
		& \ding{51}    % guava
		& \ding{51}    % simpleeventbus
		& \ding{51}    % mbassador
		& \ding{51} \\ % mycila

		& \textbf{Async. Invocation}
		& \ding{51}    % eventbus
		& \ding{51}    % guava
		& \ding{51}    % simpleeventbus
		& \ding{51}    % mbassador
		& \ding{51} \\ % mycila

		\tabucline[0.5pt]{2-7}

		\multirow{3}{*}{\textit{non-functional}}

		& \textbf{Popularity}
		& high   % eventbus
		& medium % guava
		& low    % simpleeventbus
		& medium % mbassador
		& low \\ % mycila

		& \textbf{Maturity}
		& high      % eventbus
		& medium    % guava
		& medium    % simpleeventbus
		& medium    % mbassador
		& medium \\ % mycila

		& \textbf{Documentation}
		& high      % eventbus
		& low       % guava
		& low       % simpleeventbus
		& medium    % mbassador
		& medium \\ % mycila

		\tabucline[0.5pt]{2-7}

	\end{tabu}
	\caption{Feature comparison of Java PubSub libraries.}
	\label{table:pubsub_comparison}
\endgroup

The first library we look at is EventBus.
As can be seen in \autoref{table:pubsub_comparison} on the left, EventBus supports most of the functionality we need.
From the libraries presented here it is also the oldest one, so it is mature, fairly popular and well documented.
However, outdated coding practices and many features also make it fairly heavyweight.
Guava Event Bus on the other hand is a rather simple PubSub library.
It is part of the Google core libraries for Java 1.6+ and is therefore fairly popular, but it lacks in documentation.
It also does not support weak references and filtering, which does not make it a good fit for our use case.

Simple Java Event Bus is a simpler alternative to EventBus.
It lacks some of the advanced features of EventBus but is also easier to use.
Compared to the other libraries it is not that popular and lacks in documentation.
MBassador is a light-weight and performance minded PubSub library.
As we can see in \autoref{table:pubsub_comparison}, it supports all functional features that we need and some more.
It is also relatively mature, has good enough documentation and is somewhat popular.
Finally, we have Mycila PubSub, a modern replacement for EventBus.
It supports all the functional features we need, except filtering.
Its documentation is good enough, but because it is relatively new, it is not very popular yet and may lack in maturity.

From the alternatives presented here, MBassador seems to be the only one that offers all the functionality we need combined with relative maturity and good documentation.
We will therefore use it for our implementation.
