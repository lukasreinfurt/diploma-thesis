\subsection{State Machine Libraries}
\label{implementation:selecting:statemachine}

Because we want to implement the bootware process with a FSM, we must now decide how we will do it.
It would certainly be possible to go with a hand made state machine implementation, but the time for this diploma thesis is limited and we should use it for the actual design of the bootware.
Therefore, it would be better to use an existing state machine library.
In general, we are looking for an event-driven FSM, which allows us to define a set of states and transition between those states when specific events occur.
Ideally we would prefer a standardized way to define the FSM and avoid proprietary formats.
But we also do not want the FSM to be overly complex to use and want to avoid introducing additional conversion or compilation steps.
\autoref{table:statemachine_comparison} shows six state machine libraries available for Java.

Apache Commons SCXML\footref{scxml} aims to be a java state machine engine that is capable of executing state machines defined in \nom{State Chart XML}{SCXML}.
SCXML is a working draft specification for a general-purpose event-based state machine language that is currently being developed by the \nom{World Wide Web Consortium}{W3C}~\autocite{scxml}.
Apache Commons SCXML looks like a good match for our needs because it is event-based and also uses a (soon to be) standard.
But the current state of the implementation seems to be lacking because the SCXML specification has changed a lot.
The most recent release is version 0.9, which was released in late 2008.
It is about to be replaced by version 2.0 that is currently being worked on and includes major changes, but a release date is not yet in sight\footnote{\url{http://commons.apache.org/proper/commons-scxml/roadmap.html}}.

EasyFlow\footref{easyflow} is a simple and lightweight FSM for Java.
It is event-driven, but only supports describing the FSM directly in Java code.
Compared to the other alternatives, it is not very well documented and not very popular.
There also is \nom{State Machine Compiler}{SMC}\footref{smc}, a state machine compiler that targets fifteen different programming languages, including Java.
It generates FSMs from a definition in \textit{.sm} files.
SMC is mature and has good documentation, but the use of an extra definition language and the extra step of compiling it into a Java representation seem to be to complicated for our needs.

\pagebreak

\begingroup
	\centering
	\captionsetup{type=table}
	\begin{tabu}[!htbp]{rX[2l]|[0.5pt]X[c]X[c]X[c]X[c]X[c]X[c]}

		&
		& \multicolumn{6}{c}{\textit{State Machine Libraries}} \\[10pt]

		&
		& \begin{sideways} \textbf{Commons SCXML\footnote{\url{http://commons.apache.org/proper/commons-scxml/}\label{scxml}}} \end{sideways}
		& \begin{sideways} \textbf{EasyFlow\footnote{\url{https://github.com/Beh01der/EasyFlow}\label{easyflow}}} \end{sideways}
		& \begin{sideways} \textbf{SMC\footnote{\url{http://smc.sourceforge.net/}\label{smc}}} \end{sideways}
		& \begin{sideways} \textbf{stateless4j\footnote{\url{https://github.com/oxo42/stateless4j/}\label{stateless4j}}} \end{sideways}
		& \begin{sideways} \textbf{squirrel-foundation\footnote{\url{https://github.com/hekailiang/squirrel}\label{squirrel}}} \end{sideways}
		& \begin{sideways} \textbf{Unimod\footnote{\url{http://unimod.sourceforge.net/}\label{unimod}}} \end{sideways} \\

		\tabucline[0.5pt]{2-8}

		% NO  = \ding{55}
		% YES = \ding{51}

		\multirow{2}{*}{\begin{sideways}\textit{functional}\end{sideways}}

		& \textbf{Event Driven}
		& \ding{51}    % scxml
		& \ding{51}    % easyflow
		& \ding{51}    % smc
		& \ding{51}    % stateless4j
		& \ding{51}    % squirrel
		& \ding{51} \\ % unimod


		& \textbf{Description \linebreak Language}
		& SCXML                  % scxml
		& Java                   % easyflow
		& .sm                    % smc
		& Java                   % stateless4j
		& Java, SCXML            % squirrel
		& UML, \linebreak XML \\ % unimod

		\tabucline[0.5pt]{2-8}

		\multirow{4}{*}{\begin{sideways}\textit{non-func.}\end{sideways}}

		& \textbf{Complexity}
		& med.    % scxml
		& low    % easyflow
		& med.    % smc
		& low    % stateless4j
		& low    % squirrel
		& high \\ % unimod

		& \textbf{Popularity}
		& med.    % scxml
		& low       % easyflow
		& med.    % smc
		& low       % stateless4j
		& med.    % squirrel
		& med. \\ % unimod

		& \textbf{Maturity}
		& low     % scxml
		& med.  % easyflow
		& high    % smc
		& med.  % stateless4j
		& med.  % squirrel
		& high \\ % unimod

		& \textbf{Documentation}
		& med.  % scxml
		& low     % easyflow
		& high    % smc
		& low     % stateless4j
		& high    % squirrel
		& high \\ % unimod

		\tabucline[0.5pt]{2-8}

	\end{tabu}
	\caption{Feature comparison of Java state machine libraries.}
	\label{table:statemachine_comparison}
\endgroup

Stateless4j\footref{stateless4j} is a lightweight library for creating FSMs directly in Java code.
Compared to the other alternatives, it lacks in documentation and does bot seem to be very popular.
Squirrel-foundation\footref{squirrel} is a lightweight, flexible, and extensible FSM library for Java.
Although relatively new, it is feature rich, well documented and relatively popular.
It also supports some advanced features that might be useful.
For example, it supports SCXML import and export.
Finally, there is Unimod\footref{unimod}, a project that can create FSMs from UML descriptions created by an Eclipse plugin.
Unlike the other alternative, Unimod aims to create a unified methodology for application development and not just a library.
This seems to be too complex for our needs.

From the alternatives presented here, Apache Commons SCXML would be our first choice if the standard and the implementation were more mature.
However, at this point in time this is not the case.
For this diploma thesis we will use squirrel-foundation to implement the state machine.
If Apache Commons SCXML becomes a viable option in the future, replacing squirrel-foundation could be considered.
As it supports exporting the state machine as SCXML, this could be used to ease a possible transition.
