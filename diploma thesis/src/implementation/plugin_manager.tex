\section{Plugin Manager}
\label{implementation:pluginmanager}

The plugin manager is a thin wrapper class around the Apache Felix OSGi framework.
It encapsulates all the functionality required for loading and unloading OSGi plugins.
\autoref{table:pluginmanager} lists all operations offered by the plugin manager.

\vspace*{\baselineskip}
\begingroup
	\centering
	\captionsetup{type=table}
	\renewcommand{\arraystretch}{2}
	\begin{tabu}[!htbp]{X[3,r]X[2,c]X[3,c]X[4,l]}

			\multicolumn{1}{c}{\textbf{Operation}}
		& \multicolumn{1}{c}{\textbf{Input}}
		& \multicolumn{1}{c}{\textbf{Output}}
		& \multicolumn{1}{c}{\textbf{Description}} \\

		\tabucline[0.5pt]{1-4}

			PluginManager
		& -
		& PluginManager Instance
		& Initializes, configures, and starts the OSGi framework \\

			registerShared-Object
		& Object
		& -
		& Register an object that should be shared with plugins \\

			loadPlugin
		& Path
		& Plugin Instance
		& Loads the plugin at the given path \\

			unloadPlugin
		& Path
		& -
		& Unloads the plugin at the given path \\

			unloadAllPlugins
		& -
		& -
		& Unloads all loaded plugins \\

			stop
		& -
		& -
		& Unloads all plugins and stops the OSGi framework \\

	\end{tabu}
	\caption{Operations offered by the plugin manager.}
	\label{table:pluginmanager}
\endgroup

The constructor (\textit{PluginManager}) creates a new plugin manager instance.
In the background, it initializes the OSGi framework.
Part of this initialization is telling the framework which extra packages it should export.
This is necessary so that plugins can resolve their dependencies on packages that are part of the \textit{bootware core library}.
\autoref{lst:pluginmanager} shows an excerpt of the plugin manager class where we can see the extra packages that are exported in line 42-48.
Plugins have dependencies on various bootware core packages shown here, such as the exceptions and plugins package.
They also depend on some packages from the PubSub library we use, MBassador.
All these dependencies are resolved by configuring the OSGi framework to export these packages.

\vspace*{\baselineskip}
\begin{listing}[!htbp]
	\inputminted[
		label=PluginManager.java,
		frame=topline,
		linenos,
		firstnumber=39,
		frame=lines,
		tabsize=2,
		framesep=0.3cm,
		fontsize=\small
	]{java}{implementation/assets/PluginManager.java}
	\caption{Extra packages exported by the plugin manager.}
	\label{lst:pluginmanager}
\end{listing}

The \textit{registerSharedObjects} operation allows us to register any object that is part of the bootware core with the OSGi framework, so that plugins are also able to access it.
We use this to share the EventBus instance with all plugins, so that they are able to subscribe to, and also publish, events.
The \textit{loadPlugin} operation loads the plugin at the given path into the OSGi framework and returns an instance of this plugin.
The \textit{unloadPlugin} operation unloads an already loaded plugin.
The plugin manager also offers an \textit{unloadAllPlugins} operation that unloads all loaded plugins at once.
This operation is also called during the plugin manager's \textit{stop} operation, which stops the OSGi framework, which is necessary for an orderly shutdown.

All these operations are called at specific points during the state machine execution to load and unload the needed plugins.
For example, once the state machine enters the \textit{unload event plugins} state, associated with the activities shown in the top right of \autoref{image:flow_local} and \autoref{image:flow_remote}, it executes the \textit{unloadEventPlugins} method, shown in \autoref{lst:unloadeventplugins}.
As we can see in line 293, it just calls the plugin manager's \textit{unloadAllPlugins} operation, which will unload all remaining plugins (which should be only event plugins at this point).
We can also see that exceptions are used to control the state machine transitions.
If all plugins are unloaded successfully and no exception is thrown, a \textit{success} event is fired (line 298), which will cause a transition in the state machine, in this case to the \textit{cleanup} state.
However, if somehow the \textit{unloadAllPlugins} operation fails, it throws a \textit{UnloadPluginsException}, which is caught and triggers a \textit{failure} event (line 296).
In this way, the result of the plugin manager operations can influence the execution flow of the bootware.

\vspace*{\baselineskip}
\begin{listing}[!htbp]
	\inputminted[
		label=AbstractStateMachine.java,
		frame=topline,
		linenos,
		firstnumber=287,
		frame=lines,
		tabsize=2,
		framesep=0.3cm,
		fontsize=\small
	]{java}{implementation/assets/unloadEventPlugins.java}
	\caption{The \textit{unloadEventPlugins} method defined in the \textit{AbstractStateMachine} class.}
	\label{lst:unloadeventplugins}
\end{listing}
