\section{Modeler Integration}
\label{implementation:modeler_integration}

In this section we describe the integration between the SimTech SWfMS and the bootware.
Currently, what happens is that if a workflow is is ready and should be executed, the user clicks on a button and the workflow is deployed and executed on the SimTech SWfMS.
Now we have to find a way to integrate the bootware into this process.
The button is realized by an Eclipse plugin that adds SimTech specific functionality to the Modeler (which is based on Eclipse).
We therefore also have to create some kind of Eclipse plugin to hook into this process.
There are two scenarios how we could go about this.

We could extend the existing plugin with the functionality that we need for the bootware.
In this case, we would always load the bootware extensions in the Modeler, even if we don't use the bootware at all.
We could also use a feature called extension points.
Eclipse plugins can declare extensions points, which allow other plugins to extend or customize parts of the plugin\footnote{\url{http://wiki.eclipse.org/FAQ_What_are_extensions_and_extension_points\%3F}}.
We could define an extension point in the already existing eclipse plugin and create a second plugin which implements this extension point.
This way we can separate the bootware functionality from the other SimTech extensions and keep the changes to the existing plugin to a minimum.
If a user doesn't need the bootware functionality, he doesn't have to load the bootware plugin and the SimTech plugin will continue to function as before.

The second scenario looks preferable to the first one, so this is what we are going to do.
We modify the already existing Eclipse plugin with an extension point that is triggered at the beginning of the existing deployment process.
If the bootware plugin is loaded into the Modeler, it will implement this extension point and set up the SimTech SWfMS before the already existing deployment code continues.
If it is not loaded, nothing new will happen and the existing deployment code will be executed like before.
The bootware plugin can also add additional extension to the modeler, for example a configuration dialog for setting up the context or a view that shows progress messages from the bootstrapping process.

\textcolor{red}{describe packaging. folder drop in to eclipse plugin dir. containing local bootware. remote bootware as payload}
