\section{Web Service Interface}
\label{implementation:webservice}

In \autoref{design:communication} we decided to use web service calls and returns as external communication mechanism and in \autoref{design:context} we decided to pass along a context.
Now, we need to the define the interface that will be made available by the web service to the outside.
We obviously need the two main operations, deploy and undeploy, to be available from the outside.
In \autoref{design:context} we also described the \textit{setConfiguration} operation that has to be supported.

\subsection{Deploy}

The \textit{deploy} operation will be called by at least two different components.
Once by the bootware modeler plugin to deploy the remote bootware and the workflow middleware, and then each time the provisioning manager needs to provision a new service during a workflow execution.
\autoref{lst:webservice:deployrequest} shows an exemplary deploy request as soap message.
In line 6 we can see that the deploy method is called with the context provided as argument in line 7-11.
In this particular example, only the required plugins are specified, which could be a call from the provisioning manager.

\vspace*{\baselineskip}
\begin{listing}[!htbp]
	\inputminted[
		label=deploy-request.xml,
		frame=topline,
		linenos,
		frame=lines,
		tabsize=2,
		framesep=0.3cm,
		fontsize=\small
	]{xml}{implementation/assets/deploy-request.xml}
	\caption{Sample deploy request in a soap message.}
	\label{lst:webservice:deployrequest}
\end{listing}

The response that is return once the request has been executed successfully is shown in \autoref{lst:webservice:deployresponse}.
It contains a list of endpoint references in line 5-10, which contains a reference to the payload that was deployed during the request, in this case OpenTOSCA.

\vspace*{\baselineskip}
\begin{listing}[!htbp]
	\inputminted[
		label=deploy-response.xml,
		frame=topline,
		linenos,
		frame=lines,
		tabsize=2,
		framesep=0.3cm,
		fontsize=\small
	]{xml}{implementation/assets/deploy-response.xml}
	\caption{Sample deploy response in a soap message.}
	\label{lst:webservice:deployresponse}
\end{listing}

If the deploy request somehow failed, a soap message containing a soap fault will be returned, which is shown in \autoref{lst:webservice:deployerror}.
It contains a fault string with an error description in line 5, as well as the original DeployException that was thrown by the deploy operation in line 7-10.

\vspace*{\baselineskip}
\begin{listing}[!htbp]
	\inputminted[
		label=deploy-error.xml,
		frame=topline,
		linenos,
		frame=lines,
		tabsize=2,
		framesep=0.3cm,
		fontsize=\small
	]{xml}{implementation/assets/deploy-error.xml}
	\caption{Sample deploy error in a soap message.}
	\label{lst:webservice:deployerror}
\end{listing}

\subsection{Undeploy}

Like the \textit{deploy} operation, the \textit{undeploy} operation will be called by multiple components to reverse the actions that where previously made by deploy operations.
\autoref{lst:webservice:undeployrequest} shows an exemplary undeploy request in a soap message.
As argument it contains one or more endpoint references to already deployed payloads, as can be seen in line 7-12.

\vspace*{\baselineskip}
\begin{listing}[!htbp]
	\inputminted[
		label=undeploy-request.xml,
		frame=topline,
		linenos,
		frame=lines,
		tabsize=2,
		framesep=0.3cm,
		fontsize=\small
	]{xml}{implementation/assets/undeploy-request.xml}
	\caption{Sample undeploy request in a soap message.}
	\label{lst:webservice:undeployrequest}
\end{listing}

When all payloads have been undeployed successfully, a response will be send, as shown in \autoref{lst:webservice:undeployresponse}.
The response is empty since there is nothing interesting to return.

\vspace*{\baselineskip}
\begin{listing}[!htbp]
	\inputminted[
		label=undeploy-response.xml,
		frame=topline,
		linenos,
		frame=lines,
		tabsize=2,
		framesep=0.3cm,
		fontsize=\small
	]{xml}{implementation/assets/undeploy-response.xml}
	\caption{Sample undeploy response in a soap message.}
	\label{lst:webservice:undeployresponse}
\end{listing}

In case of a failure, an error will be return.
As can be seen in \autoref{lst:webservice:undeployerror}, it has the same layout as the error returned by the deploy operation.
It contains a soap fault string in line 5 and the original UndeployException thrown by the undeploy operation in line 7-10.

\vspace*{\baselineskip}
\begin{listing}[!htbp]
	\inputminted[
		label=undeploy-error.xml,
		frame=topline,
		linenos,
		frame=lines,
		tabsize=2,
		framesep=0.3cm,
		fontsize=\small
	]{xml}{implementation/assets/undeploy-error.xml}
	\caption{Sample undeploy error in a soap message.}
	\label{lst:webservice:undeployerror}
\end{listing}

\subsection{SetConfiguration}

In addition to the main deploy and undeploy operations, the bootware web service also supports the \textit{setConfiguration} operation.
Using this operation, the configuration can be set independently from deploy requests if necessary.
\autoref{lst:webservice:setconfigurationrequest} shows an exemplary setConfiguration request.
In line 7-23, it contains a configuration list, which can contain one or more configuration sets.
Each configuration set is made up of one or more configuration entries, which are key value pairs, where the key describes the configuration type and the value the actual configuration value.
What content a particular key has to contain depends on what the plugins are looking for when they read the configuration.
In the example code in line 9, we send one configuration set for AWS, which consists of two credentials, a secret key in line 12-15 and an accessKey in line 16-19.

\vspace*{\baselineskip}
\begin{listing}[!htbp]
	\inputminted[
		label=setConfiguration-request.xml,
		frame=topline,
		linenos,
		frame=lines,
		tabsize=2,
		framesep=0.3cm,
		fontsize=\small
	]{xml}{implementation/assets/setConfiguration-request.xml}
	\caption{Sample setConfiguration request in a soap message.}
	\label{lst:webservice:setconfigurationrequest}
\end{listing}

If the setConfiguration operation was successful, the response in \autoref{lst:webservice:setconfigurationresponse} will be returned.
Again, it is empty, since there is nothing interesting to return.

\vspace*{\baselineskip}
\begin{listing}[!htbp]
	\inputminted[
		label=setConfiguration-response.xml,
		frame=topline,
		linenos,
		frame=lines,
		tabsize=2,
		framesep=0.3cm,
		fontsize=\small
	]{xml}{implementation/assets/setConfiguration-response.xml}
	\caption{Sample setConfiguration response in a soap message.}
	\label{lst:webservice:setconfigurationresponse}
\end{listing}

Like the deploy and undeploy operations, the setConfiguration operation also returns an error message if the operation failed.
As can be seen in \autoref{lst:webservice:setconfigurationerror}, it also contains a soap fault string in line 5 and the original SetConfigurationException thrown by the setConfiguration operation in line 7-10.

\vspace*{\baselineskip}
\begin{listing}[!htbp]
	\inputminted[
		label=setConfiguration-error.xml,
		frame=topline,
		linenos,
		frame=lines,
		tabsize=2,
		framesep=0.3cm,
		fontsize=\small
	]{xml}{implementation/assets/setConfiguration-error.xml}
	\caption{Sample setConfiguration error in a soap message.}
	\label{lst:webservice:setconfigurationerror}
\end{listing}

\subsection{GetActivePayloads}

The getActivePayloads operation is called by the provisioning manager to retrieve already deployed provisioning engines.
If a provisioning engine it needs is already active, it doesn't have to call the bootware to provision a new one.
\autoref{lst:webservice:getactivepayloadsrequest} shows a getActivePayloads request in a SOAP message.
No parameters are required.

\vspace*{\baselineskip}
\begin{listing}[!htbp]
	\inputminted[
		label=getActivePayloads-request.xml,
		frame=topline,
		linenos,
		frame=lines,
		tabsize=2,
		framesep=0.3cm,
		fontsize=\small
	]{xml}{implementation/assets/getActivePayloads-request.xml}
	\caption{Sample getActivePayloads request in a soap message.}
	\label{lst:webservice:getactivepayloadsrequest}
\end{listing}

The response that is returned contains a list of all payloads that where active when the request was made.
As we can see in \autoref{lst:webservice:getactivepayloadsresponse} lines 6-11, it contains a payloads list with zero or more entries.
Each entry consists of a key/value pair, where the key identifies the payload and the value contains a URL to the payload.
In this example, the entry points to an OpenTOSCA container instance.

\vspace*{\baselineskip}
\begin{listing}[!htbp]
	\inputminted[
		label=getActivePayloads-response.xml,
		frame=topline,
		linenos,
		frame=lines,
		tabsize=2,
		framesep=0.3cm,
		fontsize=\small
	]{xml}{implementation/assets/getActivePayloads-response.xml}
	\caption{Sample getActivePayloads response in a soap message.}
	\label{lst:webservice:getactivepayloadsresponse}
\end{listing}

If the getActivePayloads request failed for some reason, an error message is returned.
As can be seen in \autoref{lst:webservice:getactivepayloadserror}, it contains a soap fault string in line 5 and the original GetActivePayloadsException thrown by the getActivePayloads operation in line 7-10.

\vspace*{\baselineskip}
\begin{listing}[!htbp]
	\inputminted[
		label=getActivePayloads-error.xml,
		frame=topline,
		linenos,
		frame=lines,
		tabsize=2,
		framesep=0.3cm,
		fontsize=\small
	]{xml}{implementation/assets/getActivePayloads-error.xml}
	\caption{Sample getActivePayloads error in a soap message.}
	\label{lst:webservice:getactivepayloadserror}
\end{listing}

\subsection{Shutdown}

The shutdown operation triggers the shutdown procedure.
During this procedure, all active payloads will be undeployed.
The local bootware will also forward this request to the remote bootware and wait for a response so that it can deprovision the remote bootware before shutting down itself.
\autoref{lst:webservice:shutdownrequest} shows a shutdown request in a SOAP message.
No parameters are required.

\vspace*{\baselineskip}
\begin{listing}[!htbp]
	\inputminted[
		label=shutdown-request.xml,
		frame=topline,
		linenos,
		frame=lines,
		tabsize=2,
		framesep=0.3cm,
		fontsize=\small
	]{xml}{implementation/assets/shutdown-request.xml}
	\caption{Sample shutdown request in a soap message.}
	\label{lst:webservice:shutdownrequest}
\end{listing}

If the additional processes executed during shutdown (i.e. undeploy payloads or middleware) were successful, the response in \autoref{lst:webservice:shutdownresponse} will be returned.

\vspace*{\baselineskip}
\begin{listing}[!htbp]
	\inputminted[
		label=shutdown-response.xml,
		frame=topline,
		linenos,
		frame=lines,
		tabsize=2,
		framesep=0.3cm,
		fontsize=\small
	]{xml}{implementation/assets/shutdown-response.xml}
	\caption{Sample shutdown response in a soap message.}
	\label{lst:webservice:shutdownresponse}
\end{listing}

If the additional processes failed for some reason, an error response like the one shown in \autoref{lst:webservice:shutdownerror} will be returned.
It contains a soap fault string in line 5 and the original ShutdownException thrown by the shutdown operation in line 7-10.

\vspace*{\baselineskip}
\begin{listing}[!htbp]
	\inputminted[
		label=shutdown-error.xml,
		frame=topline,
		linenos,
		frame=lines,
		tabsize=2,
		framesep=0.3cm,
		fontsize=\small
	]{xml}{implementation/assets/shutdown-error.xml}
	\caption{Sample shutdown error in a soap message.}
	\label{lst:webservice:shutdownerror}
\end{listing}
