\section{Bootstrapping}

The term \textit{to bootstrap sth.} appears to have originated in the early 19th century in the United States, where phrases like "pulling oneself up over a fence by the straps of one's boots" where used as a figure for an impossible task~\autocite{bootstrap:history}.
In the early 20th century the metaphor's sense shifted to suggest a possible task, where one improves one's situation by one's own efforts without help from others.
An example of this can be found in James Joyce's Ulysses from 1922, where he writes about "others who had forced their way to the top from the lowest rung by the aid of their bootstraps"~\autocite{bootstrap:ulysses}.
From there, the metaphor extended to the general meaning it has today, which is the act of starting a self-sustaining process that proceeds without help from the outside.

An early reference to bootstrapping in the context of computing dates back to 1953, describing the "bootstrapping technique" as follows: "Pushing the load button then causes one full word to be loaded into a memory address [...], after which the program control is directed to that memory address and the computer starts automatically. [...] [This] full word may, however, consist of two instructions of which one is a Copy instruction which can pull another full word [...], so that one can rapidly build up a program loop which is capable of loading the actual operating program."~\autocite[p.~1273]{bootstrap:early}.

The term bootstrapping is also used with a similar meaning in a business context, where it refers to the process of starting and sustaining a company without outside funding.
The company is started with money from the founders which is used to develop a product that can be sold to customers.
Once the business reaches profitability it is self-sufficient and can use the profits it generates to organically grow the business further~\autocite{bootstrap:business}.

In this diploma thesis, bootstrapping describes the process of starting a simple program, that without further help is able to start much more complex programs.
These complex programms might require additional middleware, databases, or other components.
During the bootstrapping process, all these dependencies will be set up automatically.
