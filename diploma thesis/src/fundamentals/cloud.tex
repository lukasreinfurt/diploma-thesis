\section{Cloud Computing}

Cloud computing emerged in recent years as an alternative to traditional IT.
Compared to traditional IT, it offers customers far more flexibility in terms of short term access to and scalability of resource, such as servers, databases, communication service, etc.
This increased flexibility is the result of a combination of certain technologies and business models that, although having been around for a while individually, where combined only in recent years.

Since cloud computing is a relatively new phenomenon, there are many definitions of it scattered around. \citeauthor{cloud:def:towards} looked at over 20 of them and proposed the following definition:

\begin{quote}
	"Clouds are a large pool of easily usable and accessible virtualized resources (such as hardware, development platforms and/or services). These resources can be dynamically reconfigured to adjust to a variable load (scale), allowing also for an optimum resource utilization. This pool of resources is typically exploited by a pay-per-use model in which guarantees are offered by the Infrastructure Provider by means of customized SLAs."\footnote{\nom{Service Level Agreement}{SLA}: "An agreement that sets the expectations between the service provider and the customer and describes the products or services to be delivered, the single point of contact for end-user problems and the metrics by which the effectiveness of the process is monitored and approved."~\autocite{def:sla}}~\autocite{cloud:def:towards}
\end{quote}

The \nom{National Institute of Standards and Technology}{NIST} also proposes a definition:

\begin{quote}
	"Cloud computing is a model for enabling ubiquitous, convenient, on-demand network access to a shared pool of configurable computing resources (e.g., networks, servers, storage, applications, and services) that can be rapidly provisioned and released with minimal management effort or service provider interaction."~\autocite{cloud:def:nist}
\end{quote}

Today, there are many different cloud providers offering a huge selection of services.
The range of providers spans from titans like Amazon\footnote{\url{http://aws.amazon.com}}, Google\footnote{\url{https://cloud.google.com}}, Microsoft\footnote{\url{http://azure.microsoft.com}}, and IBM\footnote{\url{http://www.ibm.com/cloud-computing}} to small, focused providers like \textcolor{red}{...}
The next section shows Amazon's cloud services in more detail, since those will be used in this diploma thesis.

\subsection{Amazon Web Services}

In 2006, Amazon started offering cloud resource under the umbrella of \nom{Amazon Web Services}{AWS}.
Since then, their offerings steadily increased and do now comprise over 20 different products and services for computing, data storage, content delivery, analytics, deployment, management, and payment in the cloud~\autocite{aws:about}.
