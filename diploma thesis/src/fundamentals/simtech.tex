\section{SimTech}

Since 2005, the German federal and state government have been running the Excellence Initiative\footnote{\url{http://www.dfg.de/en/research_funding/programmes/excellence_initiative/index.html}}, which aims to promote cutting-edge research, thereby increasing the quality and international competitiveness of German universities.
In three rounds of funding, universities have competed with project proposals in three areas: Institutional Strategies, Graduate Schools, and Clusters of Excellence.
In total, the winners received approximately 3.3 billion euros since 2005, split up between 14 Instituational Strategies, 51 Graduate Schools, and 49 Clusters of Excellence~\autocite[pp.~16-18]{excellence:glance}.

\nom{Simulation Technology}{SimTech} is one of the Clusters of Excellence that are funded by the Excellence Initiative.
In a partnership between the University of Stuttgart, the German Aerospace Center, the Fraunhofer Institute for Manufacturing Engineering and Automation, and the Max Planck Institute for Intelligent Systems, it combines over 60 project from researchers in Engineering, Natural Science, and the Life and Social Sciences.
The aim of SimTech is to improve existing simulation strategies and to create new simulation solutions~\autocite[pp.~109]{excellence:glance}.

In the SimTech project, seven individual research areas collaborate in seven different project networks, one of which is project network 6: \textit{Cyber Infrastructure and Beyond}.
The goal of this project network is to build an easy-to-use infrastructure that supports scientists in their day to day work with simulations~\autocite{simtech:projectnetwork6}.

As part of this project, the SimTech \nom{Simulation Workflow Management System}{SWfMS} was developed.
It is a tool that enables scientists to easily create, manage and execute simulation workflows~\autocite{workflow:simulation:flexibility}.
