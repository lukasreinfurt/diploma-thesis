\chapter{Summary and Conclusion}
\label{conclusion}

In this diploma thesis we presented a design for a bootware system that is able to deploy various provisioning engines as well as a workflow middleware into remote environments, on-demand and fully automatic.
Starting from previous work, we compared possible architecture alternatives and selected a 2-tiered architecture consisting of generic local and remote bootware components.
We also introduced the notion of a bootware adapter to connect the local bootware component to a specific modeler.
We described a web service interface to allow various components to communicate with the bootware.
We made this architecture extensible via plugins and described five different plugin types.
We also added an event bus to allow plugin to create and react to events.
We described the execution flow that was implemented with a finite state machine.

Then, we presented details of the implementation of the bootware components and the integration into the SimTech SWfMS.
We described a specific implementation of the bootware adapter, the bootware plugin, an Eclipse plugin that integrates the bootware into the existing SimTech Modeler environment.
We explained the bootware core library that is used as foundation for both the local and remote bootware implementation.
We also selected Apache Felix to implement the plugins, MBassador for the internal event bus, and squirrel-foundation for the state machine implementation.
We described the content of the context object and the various web service requests and responses.
Finally, we gave an overview over various plugins, including a resource plugin for Amazon EC2 instances, a SSH communication plugin, an application plugin for the remote bootware, and an event plugin for file logging.

There were some aspects that we did not further elaborate on.
A plugin repository has to be created for the bootware to reach its full potential.
Communication with the bootware has to be made secure before it can be used in a real life environment.
Other improvements like better modeler integration and failure management should be considered.
These tasks are left for future work to explore.
In conclusion, there is still work to be done, but the work we presented here should have build a foundation for a part of a system that allows the SimTech SWfMS and other simulation workflow management systems to be used in a fashion that is more in line with scientific work principles.
