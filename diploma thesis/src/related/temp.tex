Several approaches for on-demand provisioning of service have been presented, but they all use ...

\citeauthor*{applyingwebservice} presented a dynamic service deployment (DSD) architecture for grid computing.
It handles service code retrieval, selects the installation location based on the result of a match making algorithm, and deploys the service at the selected location.
Their approach relies on already existing resources in the grid and does not have to provision additional infrastructure or install middleware.
It also doesn't use any existing provisioning solution for the service deployment.

\citeauthor*{ondemandbpel} describe a solution for on-demand resource provisioning for BPEL workflow activities in Amazon's EC2.
They introduce a load balancer component that is called by the workflow engine when a service call is made during a workflow execution.
If there are not enough resources available to run the requested service,
it can start new EC2 instances through an internal provisioner.
The provisioner can also be extended to support other cloud provider via the use of external configuration files.
Their approach is limited to starting and stopping preconfigured virtual machines that already contain all middleware necessary to run a service.
It is not able to create arbitrary infrastructure topologies.
It also does not rely on already existing provisioning solutions.



