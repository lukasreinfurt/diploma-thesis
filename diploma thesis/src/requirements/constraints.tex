\section{Constraints}

The bootware could theoretically be written in any major programming language but we limit our selves to Java.
The reason for this is that all the other SimTech components are written in Java, so by also using Java we fit nicely into this already existing ecosystem.
Additionally, for things like Eclipse integration we would have to use Java anyway.
We also have to keep in mind that the bootware will not be finished with this diploma thesis.
Other people will have to extend it in the future and because Java is common in general, as well as in the SimTech project, it makes sense to use it instead of another programming language.
We can further narrow our use of Java by limiting us to Java 1.6.
This also has to do with the already existing parts of the SimTech project, that are geared towards this version as well.
Using another version of Java could lead to unforeseen incompatibilities.
We also constrain the bootware usage to one bootware per user.
We do not plan for multi-tenancy, i.e. multiple users using the same bootware.
Additionally, we assume that a provisioning engine can be installed on a single computing resource, i.e. a VM.
If a particular provisioning engine requires a more complicated infrastructure topology, it can not be provisioned by the bootware in its current form.
In the next chapter we will also introduce additional constraints that became necessary during the design process and will therefore be explained at the appropriate times.
