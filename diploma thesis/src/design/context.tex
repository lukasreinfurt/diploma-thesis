\section{Context}
\label{design:context}

During the bootstrapping process, the bootware has to know certain things to be able do its job.
This information can be combined in one central object, which defines the nature of the current request: The context.
In this section we will take a closer look at this context object and its content.
How exactly the context is implemented is shown in \autoref{implementation:context}.

\begin{figure}[!htbp]
	\centering
	\includegraphics[resolution=600]{design/assets/context}
	\caption{Content of the context object.}
	\label{image:context}
\end{figure}

\autoref{image:context} shows the context object and its content.
As we can see in the upper half, it defines the plugin instances to be used for the current request.
The infrastructure plugin instance defines, which infrastructure plugin should be used to provision the requested infrastructure.
The communication plugin instance selects, which communication plugin the bootware should use to connect to this infrastructure.
The application plugin instance defines the application that should be provisioned on this infrastructure, which will be a provisioning engine in our case.
Finally, the provision workflow middleware plugin instance defines the plugin that should be used to call this provisioning engine to provision the workflow middleware.
It will use the service package reference of the workflow middleware, which is also defined in the context, as input to start the provisioning of the workflow middleware.

In the bottom half of \autoref{image:context} we can see that the context can also contain configuration for different plugins.
This is necessary because various plugins might need to be configured properly to be able to fulfill their task.
For example, most infrastructure plugins will need some kind of login credentials to authenticate with the infrastructure provider.
As another example, when creating a EC2 instance in the Amazon cloud, the user also has to select in which region this instance should be created and which ports should be opened.
These and other configuration details can be supplied from the outside with the context.
In the future, the context might be extended to hold additional information, but for this diploma thesis, this context will be sufficient.

This context has to somehow be generated from the information the bootware receives with each request.
However, components outside of the bootware should not know anything about the inner workings of the bootware, i.e. which plugins are used to fulfill a request.
All information that is supplied with a request can be seen in \autoref{image:request_context}.
The infrastructure parameter specifies, which infrastructure should be used (e.g. a specific cloud provider).
The application parameter specifies the application that should be deployed on this infrastructure (e.g. a specific provisioning engine).
The service package reference of the workflow middleware is used by the bootware to retrieve the service package to provision the workflow middleware.
In the configuration list, configuration values that have to be supplied from the outside can be specified, such as login credentials for a cloud provider.

\begin{figure}[!htbp]
	\centering
	\includegraphics[resolution=600]{design/assets/request_context}
	\caption{Information send with a request.}
	\label{image:request_context}
\end{figure}

Based on this information the bootware has to decide which specific plugin instances should be used to process the request.
In the future, this could be done by querying a registry with this information, which would then return a set of specific plugins and settings.
\autoref{image:mapping} shows an exemplary mapping of the parameters supplied in a request to the specific plugins instances and configuration.
In this example, the request wants OpenTOSCA in an Amazon cloud.
The registry would then look up, if a registry entry exists for this set of parameters.
In this case, the registry contains applications as primary keys and infrastructure as secondary keys.
The entry returned for a specific parameter combination contains the specific plugin instances and additional configuration parameters that are necessary to provision the requested application on the requested infrastructure.

\begin{figure}[!htbp]
	\centering
	\includegraphics[resolution=600]{design/assets/mapping}
	\caption{Mapping of the request paramaters to specific plugins.}
	\label{image:mapping}
\end{figure}

\textcolor{red}{aaaaaaaaaaaaaaaaaaaaaaaaaaaaaaaaaaaaaa}

Now that we have defined what the context will look like, we need to find a way to actually get it to the bootware.
There are a few things that we have to keep in mind when doing this.
First, some of these values have to be changeable by the user, for example login credentials in the configuration list, so it does not make sense to hard-code them into the bootware.
Furthermore, the lifetime of the information carried in the context varies quite a bit.
The plugin instances are only useful for one request execution and are likely to change from request to request, for example if the provisioning manager wants to provision multiple services with different provisioning engines.
Therefore, we must provide the plugin instances on a per request basis.
Note that this does not include the event plugins.
Event plugins are not changed with each request.
They are loaded once when the bootware starts and are unloaded before it stops.
Therefore, they do not belong into the context.
It makes more sense to specify the event plugins that should be loaded in a configuration file that the bootware reads on startup.

Because we already decided that we'll be using web services as external communication method, we can send the context containing the plugin types with each request as part of the soap message body.

Unlike the request plugins, the configuration might not change between requests.
Additionally, we have other components calling the bootware, which do not know (and maybe should not know) anything about some content of the configuration, like login credentials.
While this might change in the future, it would make sense to be able to set the configuration once when starting the bootware, so that it does not have to be delivered with each request and so that other components can still use the bootware without also sending a configuration.
It should however still be possible to override or update the configuration at a later point.
Overriding would allow any request to temporarily use other configuration values if necessary.
Updating the configuration at a later point could be useful, for example if the user accidentally provided the wrong credentials at the beginning.
Without this functionality, the whole bootware process could fail (even while provisioning the very last service) and would have to be started again from the beginning.
This could be avoided by providing the functionality to change the configuration even during the bootstrapping process.

For setting the configuration at the beginning and for updating it later during the process, a \textit{setConfiguration} method will be added to the bootware web service.
The configuration set by this method will be treated as the default configuration by the bootware.
It will be used during the process if no other configuration is provided.
If however a request is sent with a configuration that also contains values already set in the default configuration, these values will override already existing default values temporarily for this request.
This behavior could also be extended to other parts of the context in the future if necessary.
