\chapter{Design}
\label{design}

In this chapter we will develop the design of the bootware.
This design is held intentionally abstract.
Some specific implementation details will be described in \autoref{implementation}.
We will describe the component division, modeler integration, external and internal communication, extensibility, and other aspects of the bootware system.
We will also present a step by step description of the internal process during a bootstrapping operation, before presenting the final bootware architecture.
But before we explain these details, we present a rough overview of what we want to accomplish with the bootware and how we plan to do it.
\autoref{image:bootwaresteps} shows an overview over the steps involved in the bootstrapping process.

\begin{figure}[!htbp]
	\centering
	\includegraphics[resolution=600]{design/assets/bootware_steps}
	\caption{Overview over the steps invloved in the bootstrapping process.}
	\label{image:bootwaresteps}
\end{figure}

In the first step, a user creates a workflow in a modeler application.
Now, they want to execute the workflow, for which they need some workflow middleware (i.e. a WfMS), but at the moment, no workflow middleware is running.
So first, the bootware is started to help with setting up this middleware, as shown in step two.
The bootware can load various plugins that allow it to provision cloud resources and applications.
In step three, it uses those plugins to create some cloud resources and to deploy a provisioning engine in this environment.
In the fourth step, the bootware tells this provisioning engine to provision the workflow middleware that is needed to execute the workflow.
Then, it sets up the connection between the modeler and this middleware.
Now that the workflow middleware is running and connected, the workflow can be deployed and executed on this middleware, which is shown in step five.
During this workflow execution, various services might be called.
These services might also not be available at this time, so the workflow middleware has to call provisioning engines to provision those services.
These provisioning engines might also not exist, so the workflow middleware also calls the bootware to deploy the provisioning engines it needs to provision the services.
When the workflow execution is finished, all services, the workflow middleware, the provisioning engines, the underlying cloud resources, and the bootware are deprovisioned in the sixth and final step.

To summarize, the bootware has to be able to provision cloud resources, provisioning engines, and the workflow middleware by using various plugins.
It has to connect the local modeler to the workflow middleware and support the workflow middleware by deploying additional provisioning engines if needed.
It also has to remove all resources once the workflow execution is finished.
Now that we have a rough understanding of the bootware and the bootstrapping process, we can begin describing the various parts of its design in more detail.

\section{Component Division}
\label{design:division}

As described in \autoref{previous:ondemand}, the proposed architecture initially only envisioned one bootware component.
This architecture was expanded with the introduction of the provisioning manager, as described in \autoref{previous:dynamic}.
At this stage, the provisioning manager included all the functionality necessary to provision and deprovision provisioning engines in the cloud, in addition to the functionality already mentioned in \autoref{previous:dynamic}.
This was a somewhat convoluted design where multiple responsibilities where mixed into one component.
It was later decided that the provisioning manager should be split into two parts.
The actual provisioning manager handles the communication with the service repository and the various provisioning engines, as described before in \autoref{previous:dynamic}.
A separate bootware component handles the provisioning and deprovisioning of the provisioning engines.
At the moment, that leaves us with two bootware components, one local and one remote, where the local bootware kick-starts the remote bootware, which then handles the actual provisioning of provisioning engines.
The first question that has to be answered is whether this division is reasonable, or if another alternative makes more sense.
We will now discuss the viability of four such alternatives.

\subsection{Single Local Component}

\begin{figure}[!htbp]
	\centering
	\includegraphics[resolution=600]{design/assets/local}
	\caption{Simplified overview of the single local component architecture.}
	\label{image:local}
\end{figure}

First, we consider the simplest case: A single local bootware component as shown in \autoref{image:local}.
In this scenario, all provisioning processes are initiated from a bootware installed locally on the users machine, alongside or as part of the workflow modeler.

The advantages of this architecture lie in its simplicity.
Only one component has to be created and managed.
We would not have to deal with bringing the bootware into a cloud environment and each user would have his own personal bootware instance, so multi-tenancy would not be an issue.
There is no possible overlap in functionality, as it would be the case in a 2-tier architecture and communication between multiple bootware components does not have to be considered.

The disadvantages are caused by the component being local.
Because all the functionality is concentrated in one component, it can become quite large and complicated, which is one thing that should be avoided according to the requirements.
A much bigger problem however is the remote communication happening in this scenario.
As \autoref{image:local} shows, all calls to the bootware from the provisioning manager would leave the remote environment.
Also, all calls from the bootware to the provisioning engines would enter the remote environment.
This type of split communication can be costly and slow, as shown by \citeauthor*{cloudcmp}~\autocite{cloudcmp}.
They compared public cloud providers and measured that intra-datacenter communication can be two to three times faster and also cheaper (often free) compared to inter-datacenter communication~\autocite{cloudcmp}.

\subsection{Single Remote Component}

\begin{figure}[!htbp]
	\centering
	\includegraphics[resolution=600]{design/assets/remote}
	\caption{Simplified overview of the single remote component architecture.}
	\label{image:remote}
\end{figure}

The next obvious choice, as displayed in \autoref{image:remote}, is to put the single bootware component into a remote environment, where the disadvantages of local to remote communication would disappear.
However, this creates new problems.

Because there are not any additional components in this scenario that could manage the life-cycle of the remote bootware, the user would have to manage it by hand, which leads to two possibilities.
Either, the user provisions the bootware once in some cloud environment and then keeps this one instance running, or they provision the bootware once they need it and deprovisions it when they are done.

In the first case, the user would only have to provision the bootware once, but this creates a new problem: The user does not know where exactly to put the bootware.
Because one requirement is that multiple cloud environments should be supported, it is possible that the bootware is not located anywhere near the cloud environment where it should provision further components.
The communication problem of the single local bootware component can still occur in these cases.
While the other approaches presented here do not completely eliminate this problem, they at least have the option to move the bootware with each individual bootware execution, while in this first case, the bootware would stay in one place for multiple, possibly many bootware executions.

Another problem in this first case is that the bootware would be running all the time, even if the user does not need it, which would increase costs.
This problem could be reduced if this bootware instance is shared with others to assure a more balanced load.
But then the user would have to manage some sort of load balancing and the bootware would have to support multi-tenancy or be stateless to be able to cope with potential high usage spikes.
This would further complicate the design and implementation of the bootware and possibly increase the running costs.

In the second case, the user would provision the bootware whenever they need it. Now the user would be able to pick a cloud environment that is close to the other components that they plan to provision later.
This eliminates the two major problems of the first case but increases the effort that the user has to put into a task that they should not have to do in the first place.
Life-cycle management of the bootware should be automated completely and hidden away from the user.
Therefor, this scenario is not appropriate for our case.

\subsection{2-Tier Architecture}
\label{design:division:2tier}

\begin{figure}[!htbp]
	\centering
	\includegraphics[resolution=600]{design/assets/2_tier}
	\caption{Simplified overview of the 2-tier architecture.}
	\label{image:2_tier}
\end{figure}

Next, we take a look at a 2-tier architecture, as shown in \autoref{image:2_tier}, where the bootware is divided into two components.
On the local side we have a small and simple component which has mainly one function: To provision the larger second part of the bootware in a remote environment, near to the environment where other components will be provisioned later.

\pagebreak

This eliminates the problems of a single local or remote bootware component.
The user no longer has to be involved in the management of the remote bootware, because the local bootware handles all that.
Because we provision the remote bootware on demand, we now also can position the remote bootware close to other remote components to minimize local/remote communication and the problems resulting of it.
We can now keep the local part as simple as possible and make the remote part as complicated as it has to be.

But we also introduce new problems.
For one, we now have duplicate functionality between the two components.
Both have to know how to provision a component into multiple cloud environments.
The local bootware has to be able to put its remote counterpart into any cloud environment.
The remote bootware has to be able to provision other components into the same environment in which it runs (ideally, to minimize costs).
Because it can be located in any cloud environment, it has to be able to do this in any cloud environment.
Independent from this, it also has to be able to provision to any environment that the user or the service package chooses.
But this problem can be solved by using a plugin architecture, which allows both components to use the same plugins.
We discuss plugins in detail in \autoref{design:extensibility}.
A second problem which we cannot avoid but can solve is the communication which is now necessary between the different parts of the bootware.
More on this in \autoref{design:communication}

\subsection{Cloning}

\begin{figure}[!htbp]
	\centering
	\includegraphics[resolution=600]{design/assets/clone}
	\caption{Simplified overview of the cloned component architecture.}
	\label{image:clone}
\end{figure}

This architecture can be seen as an alternative form of the 2-tier architecture described in \autoref{design:division:2tier}.
In this case, there are also two bootwares working together and the remote bootware does most of the work.
However, the local and the remote bootware are identical, as shown in \autoref{image:clone}.
Instead of provisioning a bigger bootware in a remote environment, the local bootware clones itself.
Compared to the 2-tier architecture described before, this has the advantage that only one component has to be designed and implemented.
Duplication of any functionality would therefore not be an issue.
The disadvantage would be that the local bootware would be exactly as complex as the remote bootware and might contain functionality that it would not require for local operation and vice versa.
However, because we want to keep the whole bootware, including the remote part, fairly lightweight, it is unlikely that the complexity of the remote bootware will reach such heights that it could not be run on an average local machine.
In this case, the advantage of only having to design and implement one component seems to outweigh the disadvantage of a slightly more complex local component (compared to the 2-tier variant).
Of course, this architecture makes only sense if the functionality of the two separate components in the 2-tier architecture turns out to be mostly identical.
Therefore, we cannot decide yet if this architecture should be used.

\subsection{Decision}

Of the four alternative presented here, alternative three - the 2-tier architecture - makes the most sense.
Therefore, it is selected as the alternative of choice and used for further discussion.
We do however retain the option to transform it into alternative four if we discover that both components share much of same functionality.But this can only be judged at a later stage, when we know exactly how the internal functionality of the bootware will work.

\section{Modeler Integration}
\label{design:modeler_integration}

Looking at \autoref{image:2_tier}, we can see that the first interaction with the bootware is the call from the Modeler to the local bootware, which starts the bootstrapping process.
So in this section we are going to take a look at the integration between modeler and bootware in more detail.
The first question we face is: Why even divide the modeler and the local bootware?
Why not integrate the local bootware functionality into the modeler?
We go this route because we want the bootware to be as generic as possible.
The modeler in \autoref{image:2_tier} is not a specific modeler and in theory it should be possible to use the bootware with any modeler (and any workflow middleware) without too much modification.
So, by keeping the bootware as a separate generic component and only implementing a small, modeler specific adapter, we are able to support different environments without changing the core bootware components.
We call this abstract concept the bootware adapter, as shown in \autoref{image:modeler_plugin}.

In \autoref{requirements} we mentioned that the bootware should hook into the already existing deploy process in the modeler.
How this deployment process works depends on the actual modeler that is used, so at the moment, we can not say how exactly we can integrate in this process.
Specific integration details for the modeler used in this diploma thesis, the SimTech Modeler, will be discussed in \autoref{implementation:modeler_integration}.
We know however what needs to happen in the bootware adapter to get the bootstrapping process going.

\begin{figure}[!htbp]
	\centering
	\includegraphics[resolution=600]{design/assets/modeler_plugin}
	\caption{Modeler integration with a plugin.}
	\label{image:modeler_plugin}
\end{figure}

First, the bootware adapter has to start the local bootware so that it will be in a state where it can receive and process requests.
This is shown in \autoref{image:modeler_plugin} as deployment operation from the bootware adapter to the local bootware and involves starting an executable and maybe passing along some sort of configuration file.
Once the local bootware is running, the bootware adapter has to set up the context for the following requests.
This includes telling the bootware configuration details, like the credentials for all cloud providers that will be used.
Once this is done, the modeler has to make one request to the local bootware, containing the cloud provider, the provisioning engine, and the service package reference for the workflow middleware, which is shown in \autoref{image:modeler_plugin} as function call from the bootware adapter to the local bootware.
The local bootware will take this information and provision the remote bootware, which in turn will deploy a provisioning engine in the specified cloud environment
This provisioning engine will then provision the workflow middleware.
If successful, it returns the endpoint references of the workflow middleware to the remote bootware, which passes it back to the local bootware.
The local bootware passes it to the bootware adapter, which then has to set up the modeler to use these references for the actual workflow deployment.

This is the minimal work the bootware adapter has to do to kick off the bootstrapping process.
Additional functionality can be implemented if desired, but is not necessary for the core bootstrapping process.
This additional functionality could include user interface integration, additional bootware management functionality, etc.
The function call in \autoref{image:modeler_plugin} assumes that there exists some interface in the local bootware that is accessible from the outside.
In the next section we will discuss how this external communication mechanism will be implemented.

\subsection{External Communication}
\label{design:communication}

Since other components will have to call the bootware and since we will use a 2-tiered approach for the bootware component, we now have to decide, how the external communication with the bootware will work.
There are several factors that impact this decision.
Communication between the components should be as simple as possible, but has to support some critical features.
To keep it simple, it would make sense to use the same communication mechanism for communication between the bootware components as well as with other external components, like the ESB.

Since the provisioning processes kicked off by the bootware can potentially take a long time to finish (in the range of minutes to hours), asynchronous communication should be used between the components to avoid timeouts and blocking resources.
For the same reason, there should be some mechanism to get feedback on the current status during a long running provisioning process.

The communication with the bootware components will contain sensitive data, for example login information for cloud providers.
This information has to be provided from the outside on a call to call basis and should be transported securely to prevent malicious or fraudulent attacks.
The selected communication method therefore has to support some sort of security mechanism, ideally end-to-end encryption.
While these security mechanisms will not be used in this thesis due to time constraints, selecting the right communication method is still critical for future development.

Java provides a package for \nom{Remote Method Invocation}{RMI}, which allows object in one Java VM to invoke methods on objects in another Java VM~\autocite{rmi}.
But since RMI is limited to Java and we might want to communicate with the bootware from a component written in another programming language, RMI doesn't seem like a good fit.
For communication between programs written in different languages we could use the \nom{Common Object Request Broker Architecture}{CORBA}, a standard defined by the \nom{Object Management Group}{OMG}.
It supports mappings for common programming languages, like Java, C++, Python, and others.
Corba also supports asynchronous method invocation via callbacks~\autocite{corba:async} and transport layer encryption and other security features~\autocite{corba:security}.
Another alternative are web services via \nom{Simple Object Access Protocol}{SOAP} or \nom{Representational State Transfer}{REST}.
Like CORBA, web services also support asynchronous invocation, as well as security mechanisms~\autocite{ws:security}.

Since the whole SimTech SWfMS already uses SOAP based web service, it would make sense to also use SOAP based web services as external communication mechanism for the bootware component.
The technology and knowledge is already in place and introducing a second mechanism like CORBA would unnecessarily increase the complexity of the project, especially since CORBA doesn't offer any significant advantages over SOAP based web services.
\autoref{image:webservice} shows the addition of asynchronous web service call and return communication to the proposed architecture.

\begin{figure}[!htbp]
	\centering
	\includegraphics[resolution=600]{design/assets/simple_webservice}
	\caption{Simplified overview of the 2-tier architecture with asynchronous web service communication}
	\label{image:webservice}
\end{figure}

With asynchronous communication, long running provisioning processes won't pose a problem.
We do however still need information during those long running processes to give the user some feedback.
This can't be accomplished by the simple web service request/response pattern.
For this, a secondary communication mechanism which supports sending multiple feedback messages has to be used.

Since it is not necessary for the successful use of the bootware it would make sense to implement this secondary communication mechanism as a plugin.
This would allow us to add arbitrary communication plugins to the bootware depending on future needs.
These secondary communication channels could take any form, but a natural choice for publishing the intermediary state of the bootware would be a message queue system.
In this case, the remote bootware component pushes messages to a message queue to which the local bootware component (and other components if needs be) can subscribe to receive future messages.
\autoref{image:queue} shows the proposed architecture with an additional message queue.

\begin{figure}[!htbp]
	\centering
	\includegraphics[resolution=600]{design/assets/simple_queue}
	\caption{Simplified overview of the 2-tier architecture with asynchronous web service and a messaging queue communication}
	\label{image:queue}
\end{figure}

\subsection{Extensibility}
\label{design:extensibility}

The requirements for the bootware component state that support for different cloud environments and provisioning engines should be achieved through means of software engineering.
This requirement is intentionally vague to allow to select a fitting extension mechanism during the design process.

\subsubsection{Extension Mechanisms}

\textcolor{red}{other mechanisms?}

One possibility that would satisfy this criteria is to design interfaces for all extension points of the bootware component.
New cloud environments, provisioning engines, or other extension could then implement these interfaces.
The disadvantage of this approach is that the whole bootware component has to be recompiled, redistributed and redeployed if one extension is added or changed.

To remove this disadvantage, a more flexible architecture is needed, for example a plugin architecture. (\textcolor{red}{pattern?})
Interfaces for the extension points still exist but the extension are no longer part of the main bootware component.
They are compiled separately into plugins that can be loaded into the main bootware component on the fly.
There are several possibilities to realize such an architecture.

It is certainly possible to implement a plugin framework from scratch.
An advantage of this approach would be that the design of the plugin architecture could be tailored to our use case and would be as simple or complex as needed.
But there are also several disadvantages.
For one, we would reinvent the wheel, since multiple such frameworks already exist.
It would also shift resources away from the actual goal of this thesis, which is designing the bootware component.
Furthermore it would require a deep understanding of the language used for the implementation (\textcolor{red}{in this case Java}), which is not necessarily given.
Therefore it seems more reasonable to use one of the already existing plugin frameworks.Three such alternatives will be compared next. (\textcolor{red}{more?})

\subsubsection{Plugin Frameworks}

\begingroup
	\centering
	\captionsetup{type=table}
	\begin{tabu}[!htbp]{rl|[0.5pt]ccc}

		&
		& \multicolumn{3}{c}{\textit{Plugin Frameworks}} \\

		&
		& \begin{sideways} \textbf{JSPF\footnote{\url{https://code.google.com/p/jspf/}\label{jspf}}} \end{sideways}
		& \begin{sideways} \textbf{JPF\footnote{\url{http://jpf.sourceforge.net/}\label{jpf}}} \end{sideways}
		& \begin{sideways} \textbf{OSGi\footnote{\url{http://www.osgi.org/}\label{osgi}}} \end{sideways} \\

		\tabucline[0.5pt]{2-5}

		\multirow{7}{*}{\textit{Features}}

		& \textbf{Security}
		& \ding{55}    % jspf
		& \ding{55}    % jpf
		& \ding{51} \\ % osgi

		& \textbf{Dynamic Loading}
		& \ding{55}    % jspf
		& \ding{51}    % jpf
		& \ding{51} \\ % osgi

		& \textbf{Complexity}
		& low     % jspf
		& medium  % jpf
		& high \\ % osgi

		& \textbf{Active Development}
		& \ding{55}    % jspf
		& \ding{55}    % jpf
		& \ding{51} \\ % osgi

		& \textbf{Popularity}
		& low     % jspf
		& low     % jpf
		& high \\ % osgi

		& \textbf{Standard}
		& \ding{55}    % jspf
		& \ding{55}    % jpf
		& \ding{51} \\ % osgi

		& \textbf{Used in SimTech}
		& \ding{55}    % jspf
		& \ding{55}    % jpf
		& \ding{51} \\ % osgi

		\tabucline[0.5pt]{2-5}

	\end{tabu}
	\caption{Feature comparison of Java plugin frameworks}
	\label{table:plugin_comparison}
\endgroup

All of the frameworks that we compare here offer the basic functionality that we need to extend the core bootloader component, i.e. the developer defines interfaces that then are implemented by one or more plugins.
These plugins are compiled separately from the main component and are then packaged in \textit{.jar} files for distribution.
These packages are loaded during runtime and provide the implementation for the specific interface they implement.
There are however some advanced functional differences and some non-functional differences that will be considered here.

Dynamic loading allows us to load and replace plugins during runtime, without completely restarting the application.
While we don't know for certain if dynamic loading is needed in our case, it's one of the advanced features that might be nice to have in the future.

Security is a must have feature but is out of the scope of this thesis.
Consider the following scenario: The bootware component is used by multiple separate users who can share plugins using a plugin repository.
Without security features, a malicious user could upload a plugin to this repository which, in theory, could contain any code.
Therefore it's important to select the right framework now, so that security features can be implemented in the future.

Some non-functional features should also be considered, such as complexity, popularity, and if the framework is still in active development.

\nom{Java Simple Plugin Framework}{JSPF}\footref{jspf} is a plugin framework build for small to medium sized projects.
Its main focus is simplicity.
Therefore it does not support many of the advanced features like dynamic loading or security that other solution support.
The author explicitly states that it is not intended to replace JPF or OSGi~\autocite{jspf:faq}.

\nom{Java Plugin Framework}{JPF}\footref{jpf} is an open-source plugin framework.
Compared to JSPF it supports some advanced features like dynamic loading of plugins during runtime.
It is also more popular then JSPF.
However, the last version was released in 2007.
This is not necessarily bad but might show that there will be no future development of this framework.

\nom{Open Service Gateway initiative}{OSGi}\footref{osgi} is a plugin framework standard developed by OSGi Alliance.
It provides a general-purpose Java framework that supports the deployment of extensible bundles~\autocite{osgi:spec}.

\textcolor{red}{Decision}

\subsubsection{Plugin Repository}

Now that we have introduced plugins we face new problems.
\autoref{image:plugins} shows the current architecture, where both bootware components use their own plugins.
If a plugin is added or updated, the user has to manually copy this plugin to the right folder of one or both of the bootware components.
Furthermore, if both components use the same plugins, which they will (cloud plugins), we will have duplicate plugins scattered around.
This is inefficient, probably annoying for the user and possibly dangerous (\textcolor{red}{other world, fehler hervorrufend}) if plugins get out of sync.

\begin{figure}[!htbp]
	\centering
	\includegraphics[resolution=600]{design/assets/simple_plugins}
	\caption{Simplified overview of the 2-tier architecture with plugins}
	\label{image:plugins}
\end{figure}

To remedy this situation we introduce a central plugin repository, as shown in \autoref{image:plugin_repository}.
This repository holds all plugins of both components so it eliminates duplicate plugins.
If plugins are added or modified it has only to be done in one place.
Plugin synchronization can happen automatically when the bootware components start, so that the user is no longer involved in plugin management.
The repository also enables easy plugin sharing, which was cumbersome earlier.

\begin{figure}[!htbp]
	\centering
	\includegraphics[resolution=600]{design/assets/simple_plugin_repository}
	\caption{Simplified overview of the 2-tier architecture with a plugin repository}
	\label{image:plugin_repository}
\end{figure}

While a central plugin repository is a sensible addition to the proposed bootware architecture, its design and implementation are out of scope of this thesis.

\section{Plugins}
\label{implementation:plugins}

Now, we will describe the implementation of a few plugins.
We implemented a resource plugin that can create and remove EC2 instances in Amazon's cloud.
We created a communication plugin that allows the bootware to connect to a remote system via SSH and then execute commands on, or upload files to this system.
We also implemented two application plugin, one for the remote bootware itself and one for OpenTOSCA.
Additionally, we created event plugins, for example a file logger plugin that logs bootware events into a text file.

\subsection{AWS EC2 Plugin}

This resource plugin allows the bootware to create and remove EC2 instances in Amazon's cloud.
It uses the AWS SDK for Java\footnote{\url{http://aws.amazon.com/sdkforjava/}} to implement this functionality.
This SDK specifies a specific set of action that have to be taken to start an EC2 instance, which we map onto the operations defined by each resource plugin (i.e.: initialize, shutdown, deploy, and undeploy, as described in \autoref{design:plugins}).
\autoref{image:awsplugin} shows a simplified overview of these actions and how they map onto the resource plugin operations.

\begin{figure}[!htbp]
	\centering
	\includegraphics[resolution=600]{implementation/assets/aws_plugin}
	\caption{The operations implemented by the AWS EC2 plugin.}
	\label{image:awsplugin}
\end{figure}

The initialize operation, shown on the left of \autoref{image:awsplugin}, which is called once when the plugin is loaded, creates a client instance, which is an object on which all the following actions will be called.
The client instance is bound to a specific AWS region, which is read from the configuration object that is passed into the initialize operation.

As we can see in the deploy operation in \autoref{image:awsplugin}, we first have to create a security group\footnote{\url{http://docs.aws.amazon.com/AWSEC2/latest/UserGuide/using-network-security.html}}.
Security groups are essentially virtual firewalls that allow or deny traffic to and from all EC2 instances associated with it.
EC2 instances have to be associated with a security group, so we have to create one.
In the next step we open all ports in this security group that we later want to use for communication.
Which ports we open is determined by reading the configuration object.
We also have to create a SSH key pair and retrieve the private key, which we later use when we connect to this EC2 instance via SSH.
In the last step we create the actual EC2 instance.
Once it is up and running, the deploy operation is finished and returns an instance object which contains the URL where the EC2 instance can be reached, as well as the private key for SSH access.

The undeploy operation reverses the deploy operation.
First, it terminates the EC2 instance.
Once the instance is stopped, the key pair and the security group that were created earlier are removed.
We do not have to close the ports we opened, because they are part of the security group and do not exist anymore once the security group is removed.
After this, the EC2 instance created earlier is successfully removed.
There are no further actions necessary during the shutdown operation, but for safety we call the undeploy operation, in case it was not called earlier.

\subsection{SSH Plugin}

This communication plugin allows the bootware to connect to a remote system via SSH.
It uses the Ganymed SSH-2 library\footnote{\url{https://code.google.com/p/ganymed-ssh-2/}}, which implements the SSH-2 protocol in Java.
\autoref{image:sshplugin} shows a simplified overview of the actions necessary to create a SSH connection and how they map onto the communication plugin operations.

\begin{figure}[!htbp]
	\centering
	\includegraphics[resolution=600]{implementation/assets/ssh_plugin}
	\caption{The operations implemented by the SSH plugin.}
	\label{image:sshplugin}
\end{figure}

No actions are taken in the initialize operation.
During the connect operation, we first have to create a connection object, which is bound to a certain host name, i.e. the IP address of the remote system that we want to connect to.
We get this address from the instance object passed into the connect operation.
Then, we have to authenticate this connection.
Multiple authentication methods are supported by SSH-2 protocol, including password and public key authentication.
The necessary values for these authentication methods are read from the instance object passed into the connect operation.
Once the connection is authenticated, a connection object is returned, which supports the execute and upload operation that other components can use.

The disconnect operation simply closes the connection associated with the connection object that is passed into it.
The disconnect operation is also called by the shutdown operation at the end of the plugin life cycle to close any connection that might still be open.

\subsection{Remote Bootware Plugin}

This application plugin allows the local bootware to install the remote bootware on a remote system.
\autoref{image:remotebootwareplugin} shows a simplified overview of the steps involved in the installation of the remote bootware and how they map onto the application plugin operations.
The undeploy and stop operations where omitted because they are not really required in this case.

\begin{figure}[!htbp]
	\centering
	\includegraphics[resolution=600]{implementation/assets/remotebootware_plugin}
	\caption{The operations implemented by the remote bootware plugin.}
	\label{image:remotebootwareplugin}
\end{figure}

In this plugin, the initialize operation does not take any actions.
The deploy operation first uses the operations provided by the connection object it receives as input to upload the remote bootware files from the local to the remote machine.
Then, it checks if the Java version required to execute the remote bootware is present.
If not, it installs the required Java version.
The remote bootware should now be ready to start.
In the start operation a command to execute the remote bootware is sent to the remote machine.
Then, the port for the remote bootware web interface is polled until a response is received, which means that the remote bootware should now be ready.
Finally, the URL to the remote bootware is returned.

\subsection{OpenTOSCA Plugin}

This application plugin allows the bootware to install an OpenTOSCA container on an EC2 instance.
It executes the installation steps described in the OpenTOSCA manual over a connection provided by a communication plugin.
\autoref{image:opentoscaplugin} shows a simplified overview of the steps involved in the installation of OpenTOSCA and how they map onto the application plugin operations.
The undeploy and stop operations where omitted because they are not really required in this case.

\begin{figure}[!htbp]
	\centering
	\includegraphics[resolution=600]{implementation/assets/opentosca_plugin}
	\caption{The operations implemented by the OpenTOSCA plugin.}
	\label{image:opentoscaplugin}
\end{figure}

The setup procedure for OpenTOSCA is very simply.
Only one command has to be executed over SSH, which will automatically download and install all necessary components.
After that, port 8080 on the EC2 instance is polled periodically until a connection is possible, which means that the installation process is finished.
The start operation only has to return the URL pointing to the OpenTOSCA instance because OpenTOSCA was already started by the installation script.

\subsection{OpenTOSCA Workflow Middleware Plugin}

This provision workflow middleware plugin allows the bootware to provision a workflow middleware using the OpenTOSCA container.
\autoref{image:opentoscamiddlewareplugin} shows a simplified overview of the steps involved in provisioning and deprovisioning the workflow middleware with OpenTOSCA and how they map onto the provision workflow middleware plugin operations.

\begin{figure}[!htbp]
	\centering
	\includegraphics[resolution=600]{implementation/assets/opentosca_middleware_plugin}
	\caption{The operations implemented by the OpenTOSCA workflow middleware plugin.}
	\label{image:opentoscamiddlewareplugin}
\end{figure}

The initialization and shutdown operations are not used in this plugin.
The provision operation first has to get the actual CSAR URL from the service package repository, for which it uses the service package reference that was passed in as parameter.
The CSAR URL is then used to upload the CSAR to the OpenTOSCA container.
Once the CSAR is uploaded, the build plan contained inside it can be executed.
The information it returns after its completion is passed back as Map<String, String> (i.e. the implementation of the information list).
The deprovision operation just executes the termination plan contained in the CSAR.

\subsection{File Logger Plugin}

This event plugin logs all events generated by the bootware to a text file.
Unlike the other plugins, it does not implement any other interface operations apart from the standard initialize and shutdown operations.
Rather, it uses event handlers to react to specific events.
\autoref{image:fileloggerplugin} shows a simplified overview of the implementation of this plugin.

\begin{figure}[!htbp]
	\centering
	\includegraphics[resolution=600]{implementation/assets/filelogger_plugin}
	\caption{The operations implemented by the file logger plugin.}
	\label{image:fileloggerplugin}
\end{figure}

The initialize operation creates a writer object which opens a text file to write into.
This writer object is then used by the two event handlers shown in the middle to write the events they receive into this file.
The event handler shown on the left reacts to all events of the type BaseEvent, which is the parent event of all events generated by the bootware.
Therefore, it logs any event generated by the bootware into the text file.
The event handler shown on the right reacts to a special DeadMessage event type generated by the PubSub library we use, MBassador.
This event is generated each time an event is published to the event bus to which no one subscribed.
Those events are not received by any listener and are therefor dead.
We log them here for debugging purposes.
The shutdown operation just closes the write object that was created by the initialize operation.

\section{Internal Communication}
\label{design:internalcomm}

We also have to consider internal communication between the bootware core and plugins, and possibly also in between plugins.
Ideally, every plugin will be able to react to events from the bootware.
These event could be triggered by the bootware core or by any plugin, but plugins should be completely independent from each other.
Since a plugin doesn't know about other plugins, it can't listen for events at other plugins directly.
The only known constant to a plugin is the bootware core.
Therefore we need a communication mechanism which allows for loosely coupled communication between the bootware core and the plugins, where plugins can register their interest for certain events with the core and also publish their own events to the core for other plugins to consume.
This essentially describes the publish-subscribe pattern~\autocite{pubsub}.

\subsection{Publish Subscribe Pattern}

The \nom{publish-subscribe pattern}{PubSub} is a messaging pattern that consists of three types of participant: An event bus (or message broker), publishers, and subscribers.
The event bus sits at the center of the communication.
He receives messages from publishers and distributes them to all subscribers that have voiced their interest in messages of a certain type by subscribing at the event bus~\autocite{pubsub}.

Using this pattern in our bootware component, we would create an event bus at the bootware core and plugins, as well as other parts of the core, could subscribe at this event bus and also publish messages through this event bus.

\begin{figure}[!htbp]
	\centering
	\includegraphics[resolution=600]{design/assets/pubsub}
	\caption{Bootware internal communication with PubSub pattern.}
	\label{image:pubsub}
\end{figure}

\subsection{Event Types}

When using PubSub and events to communicate, its usually a good idea to not only use one type of event, but many different types.
Using different kinds of event allows us to subscribe only to specific events or react differently based on the type of an event.
But what if we want to react to each event type in the same way, for example for logging purposes?
Now, many different event types complicate things more.
This is where event hierarchies become useful.
At the core of an event hierarchy is a single base event.
By extending and refining this base event, other, more specific event types can be created, which again can be used as base type for even more specific events.
This allows us to create a fine grained hierarchy of events and also enables us to subscribe to particular sub sets of this hierarchy.
This makes event handling much easier, since we can now just react to the parent event if we don't need to distinguish between different event types for a particular task.

A second mechanism to differentiate between events is some sort of severity value that each event contains.
Many events will be published in an event system, but not all of them might be of the same importance.
The majority might be of low value while a few events might be very important.
For example, for logging purposes we might not be interested in every event, but only warnings and errors.
By adding a severity attribute to the base event type, all events could be categorized in different severity groups and filtered accordingly if needed.

As we can see, we might benefit from a well thought-out event hierarchy.

\textcolor{red}{
base event
	plugin event
		plugin loading event
		plugin execution event
	state machine event
		transition event
		start event
		stop event
}

\section{Context}
\label{context}

During the bootstrapping process, the bootware has to know certain things to be able do its job.
This includes the type of a request (e.g. deploy, undeploy, etc.) and which plugins it should use to provision the infrastructure, connect to it, and deploy the payload.
This also includes login credentials for cloud providers, which are necessary to provision the infrastructure in the first place.
This information can be combined into one object which describes the whole context of the current bootstrapping process.
Hence we will call such an object the context from now on.

\vspace{1em}
\begin{listing}[!htbp]
	\inputminted[
		label=context.xml,
		frame=topline,
		linenos,
		frame=lines,
		tabsize=2,
		framesep=0.3cm,
		fontsize=\small
	]{xml}{design/assets/context.xml}
	\caption{Sample context represented in XML.}
	\label{lst:context:sample}
\end{listing}

\autoref{lst:context:sample} shows an exemplary context in XML form.
It defines

\section{Web Service Interface}
\label{design:webservice}

By now, we know that we will use a web service interface for remote communication.
\autoref{table:webserviceoperations} shows the web service operations provided by the local and remote bootware.
To trigger the basic functionality of the bootware, two operations have to be made public via the web service interface: The \textit{deploy} and the \textit{undeploy} operation.
In \autoref{design:context} we also mentioned the \textit{setConfiguration} operation for setting or updating configuration values.
We add two additional operations that we also need, the \textit{getActiveApplications} operation and the \textit{shutdown} operation.
Both the local and the remote bootware will have to implement all of these operations, except the \textit{getActiveApplications} operation, which is only needed in the remote bootware.

\vspace*{\baselineskip}
\begingroup
	\centering
	\captionsetup{type=table}
	\renewcommand{\arraystretch}{2}
	\begin{tabu}[!htbp]{rcc}

			\multicolumn{1}{c}{\textbf{Operation}}
		& \multicolumn{1}{c}{\textbf{Input}}
		& \multicolumn{1}{c}{\textbf{Success Response}} \\

		\tabucline[0.5pt]{1-3}

			deploy
		& Context
		& Information List \\

			undeploy
		& Endpoint References
		& - \\

			setConfiguration
		& Configuration List
		& - \\

			getActiveApplications\footnote{only in remote bootware}
		& -
		& Application List \\

			shutdown
		& -
		& Confirmation Message \\

	\end{tabu}
	\caption{Web service operations provided by the local and remote bootware.}
	\label{table:webserviceoperations}
\endgroup

\subsection{Deploy}

The deploy operation is called whenever a new application (e.g.: a provisioning engine, or initially, the remote bootware) should be deployed.
As input it takes a request context object as described in \autoref{design:context}.
If it was able to successfully deploy the requested application, it responds with a list of information concerning the application.
This list can contain endpoint references, ports, or any other information that might be needed later.
If the deployment failed, it responds with an error message.

\subsection{Undeploy}

The undeploy operation is essentially the reversal of the deploy operation.
As input it takes an endpoint reference to an application that should be undeploy.
If the undeployment succeeds, it responds with a success message.
If it fails, it responds with an error message.

Unlike the deploy operation it does not take a context object as input, but the context is still needed for the undeploy operation because it contains the information about which plugins have to be used.
This means that we have to store the context object used during each deploy operation so that we can retrieve it later during the corresponding undeploy operation.
This design is intentional and will be described in more detail in \autoref{design:instancestore}.

\subsection{Set Configuration}

The setConfiguration operation is used to transmit or update the default configuration used by plugins.
As input it takes a list of configurations that should be saved.
If the list provided is empty, the default configuration list saved in the bootware will be emptied.
If the list provided is not empty, the default configuration list saved in the bootware will be overwritten by this list.
The configuration can still be overwritten on a per request basis if the context send with the request also contains a configuration.
If the configuration was updated successfully, it responds with a success message.
If the configuration could not be updated, it responds with an error message.

\subsection{Get Active Applications}

The getActiveApplications operation is used by the provisioning manager to check if a provisioning engine it needs already exist.
This operation just returns a list of all active application.
There is no reason for this operation to be called on the local bootware, so this operation will be implemented in the remote bootware only.

\subsection{Shutdown}

This operation triggers the shutdown of sequence of the bootware.
It behaves a little differently in the local and remote bootware.
In the local bootware it first calls the shutdown operation of the remote bootware.
When the confirmation response from the remote bootware is received, it deprovisions all active applications that the local bootware deployed (i.e. the remote bootware).
In the remote bootware, the shutdown operation first calls a provisioning engine to deprovision the workflow middleware.
Once this is done, it deprovisions all active applications that the remote deployed (i.e. the various provisioning engines), before returning a response.

\section{Instance Store}
\label{design:instancestore}

The instance store stores information about payloads that were deployed by the bootware in the past and are still active.
In \autoref{design:webservice} we already mentioned that we need to store some information about active payloads, but we didn't explain why.
There are several reasons why this is useful.

One big reason is that we can't guarantee that an undeploy operation will be called for every payload deployed by the bootware, since we might not have control over all components that ultimately call the bootware.
We could require that for each deploy call there must eventually be an undeploy call so that everything will be cleaned up in the end, but errors can be made and it is better to have a failsafe in place.
In the worst case scenario, failing to call the undeploy operation for some payloads could lead to rogue services remaining active after a bootware execution has stopped without the user realizing it, which could get expensive.
Storing enough information allows us to undeploy remaining payloads before shutting down the bootware even if they were never explicitly undeployed.
Additionally, a warning could be return by the bootware to inform the user that some non-bootware component should be modified to explicitly undeploy all services it deployed.

Another reason to store some information about deployed payloads is to simplify the interaction with other components.
If we wouldn't store any information and make the bootware stateless, each component using the bootware (e.g. the bootware modeler plugin, the local bootware, and the provisioning manager) would be required to keep track of all payloads it deployed using the bootware, so that this information can be supplied when it's time to undeploy.
This places an extra burden on these components and scatters around the information about deployed payloads.
By storing this information in the bootware we simplify the usage of the bootware for other components and concentrate this information.

We should also think about how such a storage mechanism might be different for the local and remote bootware.
The local bootware only ever deploys the remote bootware, so here we have to keep track of only one thing.
The remote bootware on the other hand might deploy many provisioning engines during an execution.
For the local bootware it might be sufficient to store this information in a text file on the local machine where it is executed, whereas the remote bootware might use some sort of persistent storage in the cloud.
This would allow it to retrieve this information even after a crash.
For this thesis however we will be using simple in memory storage for both the local and remote bootware.
Changing that to a more sophisticated storage solution is a possible option for future improvement.

Now that we know why it makes sense to store information about active payloads, we need to discuss what exactly we need to store.
We need to store enough information to be able to undeploy an active payload without any further input.
For this we need to know: The infrastructure plugin that was used to provision the infrastructure, the connection plugin that was used to connect to it, the payload plugin that was used to deploy the active payload, and login credentials for the remote environment if necessary.
This is all contained in the context object that we used in the first place to deploy the payload, so we will just store the whole context object.
Since we also use this storage for the undeploy operation, where we get an endpoint reference as input, we have to store it in such a way that we can map a particular context object to the provided endpoint reference.

\section{Shutdown Trigger}
\label{design:shutdown}

One thing that we haven't mentioned yet is how the bootware will be shut down.
The bootware can't just stop.
It has to make sure that all payloads and all the infrastructure it has provisioned is removed before it shuts down itself.
But how does the bootware know when it's time to start this procedure?
After all, this depends on the workflow middleware.
The shutdown process should start when the workflow middleware is finished with the workflow execution, so the bootware has to be informed of this somehow.

One possibility is to trigger the shutdown procedure from the bootware plugin in the modeler.
If the bootware plugin can access this information through the modeler, it can call the shutdown operation of the local bootware, which will in turn call the shutdown operation of the remote bootware, which will eventually lead to the removal of all remote components.
If this is possible using a particular modeler depends on the modeler and the integration possibilities for the bootware plugin.

There is a second possibility that can be used instead of the first one.
We already introduced the event plugin type, which can also trigger events in the bootware, in particular the shutdown event.
An event plugin could be created that somehow communicates with the workflow middleware to receive notice when the execution is finish.
For example, in the SimTech SWfMS, the workflow engine publishes event into a message queue.
An event plugin could be created that subscribes to this messages queue and reacts to a particular event by triggering the shutdown event inside the bootware.
This plugin would then be loaded into the local bootware and would trigger the shutdown procedure, which would in turn call the shutdown operation of the remote bootware as before.

\section{Execution Flow}
\label{design:flow}

Until now we have established how the bootware can be called from outside components using a web service interface to start the bootstrapping process.
We also established that big parts of this process will be implemented as plugins.
Now, it is time to take a look at the actual internal structure of the bootware.
What follows is a step by step description of the internal process during a bootstrapping operation.

\begin{figure}[!htbp]
	\centering
	\includegraphics[resolution=600]{design/assets/flow_local}
	\caption{Execution flow in the local bootware.}
	\label{image:flow_local}
\end{figure}

\autoref{image:flow_local} shows a graph that represents the major steps during the bootware execution in the local bootware as flow diagram.
The bootstrapping process is started when the bootware adapter starts the local bootware, which is represented by the \textit{start} activity in the top left corner of \autoref{image:flow_local}.
From there, the bootware first does some initializations.
If those fail for some reason, the cleanup code will be executed before the local bootware execution is ended, as can be seen on the top right corner of \autoref{image:flow_local}.
In most cases however, the initialization should succeed.
Then, the local bootware will transition to the next activity, where it tries to load the event plugins.

The event plugins are loaded once at the beginning of the local bootware execution because they will not change at a per request basis (like the other plugins).
Any events generated by the bootware or other plugins after this point can now be handled by the event plugins.
If loading one of these plugins fails, the local bootware will try to unload already loaded plugins before continuing to the \textit{cleanup} activity.
If the plugins are loaded successfully, the local bootware transitions into the \textit{wait} activity, shown in the top center of \autoref{image:flow_local}.

The local bootware is now ready and waits for requests from the outside.
A deploy request will be send to the local bootware when the bootware adapter calls it to provision the workflow middleware.
A setConfiguration request might also be send by the bootware adapter to set the default configuration.
The undeploy and shutdown request will be triggered by the local bootware itself, when it is time to deprovision the workflow middleware and the remote bootware.

If a shutdown event is received, the local bootware will first tell the remote bootware to undeploy all active applications.
Next, the local bootware will undeploy the remote bootware by running through the undeploy process fragment shown on the bottom left with the appropriate plugins.
Then, it will shut itself down by first unloading the event plugins and then running the cleanup code.
This is the only normal way to shut down the bootware.
We only hint at the setConfiguration request here, since it just replaces the saved configuration with the one sent in the request.
Deploy and undeploy requests however are more complicated.
If such a request is received, the local bootware transitions to the next activity, where it evaluates the request context.

In the \textit{evaluate context} activity, the information send with the request is used to generate the full context, which contains all the information necessary to fulfill the request, as described in \autoref{design:context}.
If this can not be done for some reason, the local bootware returns a response containing an error message before returning to the \textit{wait} activity.
If the context is created successfully, the local bootware tries to send the request on to the remote bootware, as shown in the middle of \autoref{image:flow_local}.
For this to work, the remote bootware has to exist in the requested remote environment, which will not be the case during the first execution.
Therefore, the local bootware first has to provision the remote bootware in the requested remote environment and so it transitions to the \textit{load request plugins} activity.

In the \textit{load request plugins} activity the plugins specified in the context are loaded.
If this fails, the local bootware tries to unload already loaded plugins before returning an error response and transitioning to the \textit{wait} activity.
If the plugins are loaded successfully, the local bootware now starts either the deploy process at the bottom left, or the undeploy process, shown at the bottom right of \autoref{image:flow_local}, depending on the type of the request.

If the request was a deploy request, the local bootware will now execute the steps shown in the bottom left of \autoref{image:flow_local} one after another.
In the \textit{provision resource} activity, the deploy operation of the resource plugin will be called.
Then, in the \textit{connect} activity, a connection with this resource will be established by the communication plugin.
Over this connection, the requested application is provisioned in the \textit{provision application} activity, and started in the \textit{start application} activity, using the application plugin.
If one of these activities fails, the local bootware transitions over to the corresponding undeploy activities on the right and works its way backwards to undo all operations that where already executed.
This process fragment is the same as the undeploy process, shown on the bottom right of \autoref{image:flow_local}, which is triggered by an undeploy request.

If the \textit{stop application}, \textit{deprovision application}, or \textit{disconnect} activities fail, the local bootware just continues with the next undeploy activity because these operations are not considered critical.
However, if the \textit{deprovision resource} activity fails, the local bootware transitions to a \textit{fatal error} activity, shown at the right of \autoref{image:flow_local}, because this step is considered critical.
This activity failing could mean that resources are still active in the cloud and human interaction is necessary to remove them to stop further costs from incurring.
The \textit{fatal error} activity is responsible for taking special actions to remedy this situation.

The successful, as well as the unsuccessful execution of either the deploy or the undeploy process all finish in the \textit{unload request plugins} activity, where the plugins that where needed for this particular request are unloaded.
If everything went as planned, a remote bootware should now be running in the desired cloud environment and the local bootware can now pass on the request to this remote bootware, as shown in the center of \autoref{image:flow_local} with the \textit{send to remote} activity.
The local bootware will wait here until it receives a response from the remote bootware.

Now, we move our attention to the remote bootware, where the requests continues to be processed.
\autoref{image:flow_remote} shows the execution flow of the remote bootware.
As we can see, it is largely identical to the local bootware.
The \textit{send to remote} activity is gone because it is not needed in the remote bootware.
Instead, as the bottom of \autoref{image:flow_remote} shows, the \textit{provision middleware} and \textit{deprovision middleware} activities were added.
The remote bootware also supports the getActiveApplications request.
Other than that, the local and remote processes are the same.

\begin{figure}[!htbp]
	\centering
	\includegraphics[resolution=600]{design/assets/flow_remote}
	\caption{Execution flow in the remote bootware.}
	\label{image:flow_remote}
\end{figure}

Like the local bootware, the remote bootware went through the initialization steps shown at the top of \autoref{image:flow_remote} when it was started by the local bootware.
It then waited in the \textit{wait} activity for a request.
Now, it receives the request from the local bootware, creates the context, loads the request plugins and executes the deploy operation.
This should result in a provisioning engine being started by the application plugin.
After that, the remote bootware enters the new \textit{provision middleware} activity at the bottom left of \autoref{image:flow_remote}, which will use the just started provisioning engine to deploy the workflow middleware by executing the provision workflow middleware plugin.
Once the middleware is running, the remote bootware is finished with this request and returns the information list containing the endpoint references of the middleware in the response to the local bootware, before returning to the \textit{wait} activity.

This brings us back to \autoref{image:flow_local}, where the local bootware has now received the answer from the remote bootware in the \textit{send to remote} activity.
Now, the local bootware can finish its request by sending back a response to the bootware adapter, before returning to the \textit{wait} activity.
The local bootware is now done until it is time to undeploy the remote bootware.
Meanwhile, the bootware adapter starts the workflow execution on the middleware, during which multiple calls from the provisioning manager to the remote bootware will occur, which will each time trigger the deploy or undeploy process fragments shown at the bottom, or the getActiveApplications operation only hinted at in \autoref{image:flow_remote}.

As \autoref{image:flow_local}, \autoref{image:flow_remote} and the description above show, this is quite a complicated process with many conditional transition.
Using traditional programming methods like if/else blocks to implement this process would lead to a rather unwieldy and complicated construct with lots of nested if/else blocks.
Therefore, it could be advantageous to use other methods that are more fitting for this process.
As we already described the process as a directed graph, it would be ideal if we could take this whole graph and use it in the bootware.
Fortunately, this is possible by implementing the process using a finite state machine.

\subsection{Finite State Machine}

In theoretical computer science, a \nom{Finite State Machine}{FSM} is a formal, abstract model of computation "consisting of a set of states, a start state, an input alphabet, and a transition function that maps input symbols and current states to a next state. Computation begins in the start state with an input string. It changes to new states depending on the transition function"~\autocite{fsm}.
In this context, a state is the "condition of a finite state machine [...] at a certain time. Informally, the content of memory"~\autocite{state}.
The start state is therefore the initial condition of a FSM.
The alphabet is a "set of all possible symbols in an application. For instance, input characters used by a finite state machine, letters making up strings in a language, or symbols in a pattern element. In some cases, an alphabet may be infinite"~\autocite{alphabet}.
The transition function is a "function of the current state and input giving the next state of a finite state machine"~\autocite{transitionfn}.
FSMs can further be distinguished in deterministic and non-deterministic FSMs.
A deterministic FSM has at most one transition for each symbol and state, whereas a non-deterministic FSM can have non, one, or more transitions per symbol and state~\autocite{deterministic}.

Aside from its uses in theoretical computer science, FSMs also have practical applications in digital circuits, software applications, or as lexers in programming language compilers.
We are only interested in the use of FSMs for building software, so we can redefine what a FSM means for our case.
We want to use a FSM as an abstract machine that is defined by a finite list of states and some conditions that trigger transitions between those states.
Unlike a traditional FSM, we will not consume symbols from a set alphabet that will trigger state transitions.
We want the state transitions to be triggered by events that we can emit at any time, so we want an event-driven FSM.
The machine is in only one state at a time, its current state.
At the start of the machine execution, it will be in the start state.
From there, it can transition from one state to another when certain events are triggered, until it finally reaches an end state.
The states map directly onto the activities described in \autoref{image:flow_local} and \autoref{image:flow_remote}
When The FSM enters a state, it executes a function associated with this state, which would be the implementation of said activities.
The result of the execution of this function determines to which state the FSM will transition next.
We will talk more about the actual implementation with FSMs in \autoref{implementation}.

\section{Final Bootware Architecture}
\label{design:finalarch}

In \autoref{image:finalarch} we present the final architecture of the bootware in context with a SWfMS.
New components are marked black and include the local and remote bootware, their plugins, and the bootware adapter.
Old components that existed previously are shown in white.

\begin{figure}[!htbp]
	\centering
	\includegraphics[resolution=600]{design/assets/final_architecture}
	\caption{The final architecture of the bootware.}
	\label{image:finalarch}
\end{figure}

\autoref{image:finalarchlocal} and \autoref{image:finalarchremote} show the final architecture of the local and remote bootware.
They only differ in some small details, but this might change in the future.
At the bottom we can see some exemplary event plugins.
These are loaded at the beginning of the bootware execution by the plugin manager, shown on the left of both figures.
For demonstrations purposes, both figures shows a wider range of possible event plugins.
All these plugins provide some sort of input and/or output mechanism for the bootware.
A \nom{command-line interface}{CLI} plugin, as shown in \autoref{image:finalarchlocal}, could be used to make the bootware operations accessible via a command-line interface.
An event logger plugin could be used to write all bootware events to a log file.
We can also imagine an event queue plugin that pushes all bootware events into some message queue at the remote bootware, so that they can be consumed by other components, like the local bootware.
Finally, an undeploy trigger plugin in the local bootware, as shown in \autoref{image:finalarchlocal}, could trigger the undeployment of the bootware and all running applications by listening for a specific message at the workflow middleware.
Besides the event plugins there is always the web service interface, shown at the bottom right of both figures, which provides the standard way to interact with the bootware.

\begin{figure}[!htbp]
	\centering
	\includegraphics[resolution=600]{design/assets/final_architecture_local}
	\caption{The final architecture of the local bootware component.}
	\label{image:finalarchlocal}
\end{figure}

All event plugins and the web service interface work by implementing event handlers for certain events published at the event bus, or by publishing events to the event bus themselves.
As we can see in the center of both figures, the event bus and the state machine form the core of the bootware.
The event bus is responsible for distributing events between the various plugins and the state machine.
The state machine implements the entire bootstrapping process, as described earlier in \autoref{design:flow}.
At certain points during the bootstrapping process, operations are delegated to the plugin manager to load plugins, and to the infrastructure, communication, application, and provision workflow middleware plugins, shown at the top of both figures.

\begin{figure}[!htbp]
	\centering
	\includegraphics[resolution=600]{design/assets/final_architecture_remote}
	\caption{The final architecture of the remote bootware component.}
	\label{image:finalarchremote}
\end{figure}

The infrastructure, communication, and application plugins implement the actual bootstrapping operations.
At the top, both figures show an exemplary result of these bootstrapping operations.
In this particular case, the infrastructure plugin started a VM, to which the communication plugin set up a communication channel.
The application plugin then used this communication channel to provision the application inside the VM.
The provisioning engine plugin is only available in the remote bootware and allows it to call a provisioning engine with the details necessary to provision the workflow middleware.
This is shown in \autoref{image:finalarchremote} as an additional function call from the provision workflow middleware plugin to the previously deployed application.
During the bootstrapping procedure, events are sent from all these plugins back to the event bus to be delivered to the loaded event plugins.
As we can now see, the local and the remote bootware are quite similar, but differ in enough ways that a cloned architecture, as described in \autoref{design:division}, might not be the best choice, especially because both components might drift further apart in their functionality in the future.
Therefore, we decide to not alter our original decision to got with a 2-tiered architecture.

