\section{Web Service Interface}

Now that we know that we'll use web service calls and returns as external communication mechanism, we need to the define the interface that will be made available by the web service to the outside.
We obviously need the two main operations, deploy and undeploy, to be available from the outside.

\subsection{Deploy}

The \textit{deploy} operation will be called by at least three different components.
Once by the bootware modeler plugin to deploy the remote bootware, \textcolor{red}{once again when the provisioning engine for provisioning the workflow middleware is deployed}, and then each time the provisioning manager needs to provision a new service during a workflow execution.

\vspace*{\baselineskip}
\begin{listing}[!htbp]
	\inputminted[
		label=deploy-request.xml,
		frame=topline,
		linenos,
		frame=lines,
		tabsize=2,
		framesep=0.3cm,
		fontsize=\small
	]{xml}{design/assets/deploy-request.xml}
	\caption{Sample deploy request in a soap message.}
	\label{lst:webservice:deployrequest}
\end{listing}

\autoref{lst:webservice:deployrequest} shows an exemplary deploy request as soap message.
In line 6 we can see that the deploy method is called with the context provided as argument in line 7-11.
The response that is return once the request has been executed successfully is shown in \autoref{lst:webservice:deployresponse}.
It contains a endpoint reference element in line 6-9 which contains a reference to the payload that was deployed during the request.

\vspace*{\baselineskip}
\begin{listing}[!htbp]
	\inputminted[
		label=deploy-response.xml,
		frame=topline,
		linenos,
		frame=lines,
		tabsize=2,
		framesep=0.3cm,
		fontsize=\small
	]{xml}{design/assets/deploy-response.xml}
	\caption{Sample deploy response in a soap message.}
	\label{lst:webservice:deployresponse}
\end{listing}

\subsection{Undeploy}

Like the \textit{deploy} operation, the \textit{undeploy} operation will be called by multiple components to reverse the actions that where previously made by deploy operations.
\autoref{lst:webservice:undeployrequest} shows an exemplary undeploy request in a soap message.
As argument it contains one or more endpoint references to already deployed payloads, as can be seen in line 7-13.
When all payloads have been undeployed successfully, a response will be send to.

\vspace*{\baselineskip}
\begin{listing}[!htbp]
	\inputminted[
		label=undeploy-request.xml,
		frame=topline,
		linenos,
		frame=lines,
		tabsize=2,
		framesep=0.3cm,
		fontsize=\small
	]{xml}{design/assets/undeploy-request.xml}
	\caption{Sample undeploy request in a soap message.}
	\label{lst:webservice:undeployrequest}
\end{listing}

\subsection{SetCredentials}

In addition to the main deploy and undeploy operations, the bootware web service also supports the \textit{setCredentials} operation.
Using this operation, login credentials can be set independently from deploy requests if necessary.
\autoref{lst:webservice:setCredentialsrequest} shows an exemplary set credentials request.
In line 7-14, it contains one or more credentials as argument.
If the credentials where set successfully, a response will be returned.

\vspace*{\baselineskip}
\begin{listing}[!htbp]
	\inputminted[
		label=setCredentials-request.xml,
		frame=topline,
		linenos,
		frame=lines,
		tabsize=2,
		framesep=0.3cm,
		fontsize=\small
	]{xml}{design/assets/setCredentials-request.xml}
	\caption{Sample setCredentials request in a soap message.}
	\label{lst:webservice:setCredentialsrequest}
\end{listing}
