\section{Final Bootware Architecture}
\label{design:finalarch}

In \autoref{image:finalarch} we present the final architecture of the bootware in context with a SWfMS.
New components are marked black and include the local and remote bootware, their plugins, and the bootware adapter.
Old components that existed previously are shown in white.

\begin{figure}[!htbp]
	\centering
	\includegraphics[resolution=600]{design/assets/final_architecture}
	\caption{The final architecture of the bootware.}
	\label{image:finalarch}
\end{figure}

\autoref{image:finalarchlocal} and \autoref{image:finalarchremote} show the final architecture of the local and remote bootware.
They only differ in some small details, but this might change in the future.
At the bottom we can see some exemplary event plugins.
These are loaded at the beginning of the bootware execution by the plugin manager, shown on the left of both figures.
For demonstrations purposes, both figures shows a wider range of possible event plugins.
All these plugins provide some sort of input and/or output mechanism for the bootware.
A \nom{command-line interface}{CLI} plugin, as shown in \autoref{image:finalarchlocal}, could be used to make the bootware operations accessible via a command-line interface.
An event logger plugin could be used to write all bootware events to a log file.
We can also imagine an event queue plugin that pushes all bootware events into some message queue at the remote bootware, so that they can be consumed by other components, like the local bootware.
Finally, an undeploy trigger plugin in the local bootware, as shown in \autoref{image:finalarchlocal}, could trigger the undeployment of the bootware and all running applications by listening for a specific message at the workflow middleware.
Besides the event plugins there is always the web service interface, shown at the bottom right of both figures, which provides the standard way to interact with the bootware.

\begin{figure}[!htbp]
	\centering
	\includegraphics[resolution=600]{design/assets/final_architecture_local}
	\caption{The final architecture of the local bootware component.}
	\label{image:finalarchlocal}
\end{figure}

All event plugins and the web service interface work by implementing event handlers for certain events published at the event bus, or by publishing events to the event bus themselves.
As we can see in the center of both figures, the event bus and the state machine form the core of the bootware.
The event bus is responsible for distributing events between the various plugins and the state machine.
The state machine implements the entire bootstrapping process, as described earlier in \autoref{design:flow}.
At certain points during the bootstrapping process, operations are delegated to the plugin manager to load plugins, and to the infrastructure, communication, application, and provision workflow middleware plugins, shown at the top of both figures.

\begin{figure}[!htbp]
	\centering
	\includegraphics[resolution=600]{design/assets/final_architecture_remote}
	\caption{The final architecture of the remote bootware component.}
	\label{image:finalarchremote}
\end{figure}

The infrastructure, communication, and application plugins implement the actual bootstrapping operations.
At the top, both figures show an exemplary result of these bootstrapping operations.
In this particular case, the infrastructure plugin started a VM, to which the communication plugin set up a communication channel.
The application plugin then used this communication channel to provision the application inside the VM.
The provisioning engine plugin is only available in the remote bootware and allows it to call a provisioning engine with the details necessary to provision the workflow middleware.
This is shown in \autoref{image:finalarchremote} as an additional function call from the provision workflow middleware plugin to the previously deployed application.
During the bootstrapping procedure, events are sent from all these plugins back to the event bus to be delivered to the loaded event plugins.
As we can now see, the local and the remote bootware are quite similar, but differ in enough ways that a cloned architecture, as described in \autoref{design:division}, might not be the best choice, especially because both components might drift further apart in their functionality in the future.
Therefore, we decide to not alter our original decision to got with a 2-tiered architecture.
